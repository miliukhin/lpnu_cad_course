%xelatex
\documentclass[14pt]{extreport}

\usepackage[T2A]{fontenc}
\usepackage[english,ukrainian]{babel}
%\usepackage{fontspec}
\usepackage{amsmath}
\usepackage{mathtools}
\usepackage{caption}
\usepackage{mathspec}
\usepackage{multicol}
\setallmainfonts{Nimbus Roman}
\usepackage{graphicx}
\usepackage[a4paper,margin=0.5in]{geometry}
\usepackage{pgfplots}
\pgfplotsset{width=10cm,compat=1.8}
\usepackage{indentfirst}
\usepackage{boxedminipage}
\usepackage{wrapfig}
\usepackage{subfiles}

\begin{document}
\pagestyle{empty}

\subfile{title.tex}

%\begin{multicols}{2}
\subsection*{Переведіть числа з однієї системи числення в іншу }

\subsubsection{1. З двійкової у вісімкову 1011010011011 }
$1011010011011_2=1\cdot2^0+1\cdot2^1+0+1\cdot2^3+1\cdot2^4+
0+0+1\cdot2^7+0+1\cdot2^9+1\cdot2^{10}+0+1\cdot2^{12}=5787_{10}$

\bigskip

\begin{tabular}{c|c|c}
	5787 & 8 & 3 \\
	\hline
	723 & 8 & 3 \\
	\hline
	90 & 8 & 2 \\
	\hline
	11 & 8 & 3 \\
	\hline
	1 & 8 & 1 \\
\end{tabular}
\hspace{.1\textwidth}
$1011010011011_2=13233_8$

\subsubsection{2. З двійкової у десяткову 1011010}
$1011010_2=0+2^1+0+2^3+2^4+0+2^6=90_{10}$

\subsubsection{3. З двійкової у шістнадцяткову 1011010011010}
$1011010011010_2=0+2^1+0+2^3+2^4+0+0+2^7+0+2^9+2^{10}+0+2^{12}=
5786_{10}$

\bigskip
\begin{tabular}{c|c|c}
	5786 & 16 & 10 \\
	\hline
	361 & 16 & 9 \\
	\hline
	22 & 16 & 6 \\
	\hline
	1 & 16 & 1 \\
\end{tabular}
\hspace{.1\textwidth}
$1011010011010_2=169A_{16}$ \\

\subsubsection{4. З вісімкової  у двійкову 537 }
$537_8=7\cdot8^0+3\cdot8^1+5\cdot8^2=351_{10}$

\bigskip
\begin{tabular}{c|c|c}
	351 & 2 & 1 \\
	\hline
	175 & 2 & 1 \\
	\hline
	87 & 2 & 1 \\
	\hline
	43 & 2 & 1 \\
	\hline
	21 & 2 & 1 \\
	\hline
	10 & 2 & 0 \\
	\hline
	5 & 2 & 1 \\
	\hline
	2 & 2 & 0 \\
	\hline
	1 & 2 & 1\\
\end{tabular}
\hspace{.1\textwidth}
$537_8=101011111_2$

\subsubsection{5. З вісімкової  у десяткову 164 }
$164_8=4\cdot8^0+6\cdot8^1+1\cdot8^2=116_{10}$
\subsubsection{6. З вісімкової  у шістнадцяткову 451 }
$451_8=1\cdot8^0+5\cdot8^1+4\cdot8^2=297_{10}$

\bigskip
\begin{tabular}{c|c|c}
	297 & 16  & 9 \\
	\hline
	18 & 16 & 2 \\
	\hline
	1 & 16 & 1 \\
\end{tabular}
\hspace{.1\textwidth}
$451_8=129_{16}$

\subsubsection{7. З десяткової у двійкову 88 }
\begin{tabular}{c|c|c}
	88 & 2 & 0 \\
	\hline
	44 & 2 & 0 \\
	\hline
	22 & 2 & 0 \\
	\hline
	11 & 2 & 1 \\
	\hline
	5 & 2 & 1 \\
	\hline
	2 & 2 & 0 \\
	\hline
	1 & 2 & 1 \\
\end{tabular}
\hspace{.1\textwidth}
$88_{10}=1011000_2$
\subsubsection{8. З десяткової у вісімкову 74 }
\begin{tabular}{c|c|c}
	74&8&2\\
	\hline
	9&8&1\\
	\hline
	1&8&1\\
\end{tabular}
\hspace{.1\textwidth}
$74_{10}=112_8.$
\subsubsection{9. З десяткової у шістнадцяткову 126 }
\begin{tabular}{c|c|c}
	126 & 16 & 14\\
	\hline
	7 & 16 & 7 \\
\end{tabular}
\hspace{.1\textwidth}
$126_{10}=7E_{16}$
\subsubsection{10. З шістнадцяткової у двійкову 3C8 }
$3C8_{16}=8\cdot16^0+12\cdot16^1+3\cdot16^2=968_{10}$

\bigskip
\begin{tabular}{c|c|c}
	968 & 2 & 0 \\
	\hline
	484 & 2 & 0 \\
	\hline
	242 & 2 & 0 \\
	\hline
	121 & 2 & 1 \\
	\hline
	60 & 2 & 0 \\
	\hline
	30 & 2 & 0 \\
	\hline
	15 & 2 & 1 \\
	\hline
	7 & 2 & 1 \\
	\hline
	3 & 2 & 1 \\
	\hline
	1 & 2 & 1 \\
\end{tabular}
\hspace{.1\textwidth}
$3C8_{16}=1111001000_2$

\subsubsection{11. З шістнадцяткової у вісімкову A36 }
$A36_{16}=6\cdot16^0+3\cdot16^1+10\cdot16^2=2614_{10}$

\bigskip
\begin{tabular}{c|c|c}
	2614 & 8 & 6 \\
	\hline
	326 & 8 & 6 \\
	\hline
	40 & 8 & 0 \\
	\hline
	5 & 8 & 5 \\
\end{tabular}
\hspace{.1\textwidth}
$A36_{16}=5066_8$

\subsubsection{12. З шістнадцяткової у десяткову C9 }
$C9_{16}=12\cdot16^1+9=201_{10}$

\subsubsection{13. З двійкової у десяткову (дроби) 0.10100 }
$0.10100_2=0\cdot2^0+1\cdot2^{-1}+0+1\cdot2^{-3}+0+0=0.625_{10}$
\subsubsection{14. З десяткової у двійкову(дроби)~--- 4 дв. розр. 0.43}

$0.43\cdot2=\underline{0}.86$

$0.86\cdot2=\underline{1}.72$
\hspace{.1\textwidth}
$0.43_{10}=0.0110_2$

$0.72\cdot2=\underline{1}.44$

$0.44\cdot2=\underline{0}.88$

%--------II--------%
\subsection*{Виконати дії  і перевірку }
\subsubsection{15. Переведіть з (2) у  (8)  систему числення 11011011.1010}
Для початку переведемо число в десяткову систему численння:

$11011011.1010_2=\frac{1}{2}+\frac{1}{8}+
1+2+0+8+16+0+64+128=219\frac{5}{8}_{10}$

Тепер переведемо цілу та дробову частини у вісімкову:

\bigskip
\begin{multicols}{2}
\begin{tabular}{c|c|c}
	219 & 8 & 3 \\
	\hline
	27 & 8 & 3 \\
	\hline
	3 & 8 & 3 \\
\end{tabular}

$219_{10}=333_8$

$\frac{5}{8}\cdot8=5.0$

$219\frac{5}{8}_{10}=333.5_8$

\end{multicols}

\textbf{Перевірка:}
%$$\underbrace{110110}\underbrace{}\underbrace{1010}=333.5$$
$$333.5_8=\frac{5}{8}+3\cdot(1+8+64)=219.625$$

\begin{multicols}{2}
\begin{tabular}{c|c|c}
	219 & 2 & 1 \\
	\hline
	109 & 2 & 1 \\
	\hline
	54 & 2 & 0 \\
	\hline
	27 & 2 & 1 \\
	\hline
	13 & 2 & 1 \\
	\hline
	6 & 2 & 0 \\
	\hline
	3 & 2 & 1 \\
	\hline
	1 & 2 & 1 \\
\end{tabular}

$0.625\cdot2=\underline{1}.25$

$0.25\cdot2=\underline{0}.5$

$0.5\cdot2=\underline{1}.0$

\bigskip
Отже,
$11011011.1010_2=333.5_8$

\end{multicols}
\bigskip

\subsubsection{16. Переведіть з (2) у (10) систему числення 11011011.1010}
$11011011.1010_2=\frac{1}{2}+\frac{1}{8}+
1+2+0+8+16+0+64+128=219\frac{5}{8}_{10}$

\bigskip
\textbf{Перевірка:}
\bigskip
\begin{multicols}{2}
\begin{tabular}{c|c|c}
	219 & 2 & 1 \\
	\hline
	109 & 2 & 1 \\
	\hline
	54 & 2 & 0 \\
	\hline
	27 & 2 & 1 \\
	\hline
	13 & 2 & 1 \\
	\hline
	6 & 2 & 0 \\
	\hline
	3 & 2 & 1 \\
	\hline
	1 & 2 & 1 \\
\end{tabular}

$0.625\cdot2=\underline{1}.25$

$0.25\cdot2=\underline{0}.5$

$0.5\cdot2=\underline{1}.0$

\end{multicols}

Виконавши перевірку, переконалися, що $219\frac{5}{8}_{10}=11011011.1010_2$

\subsubsection{17. Переведіть з (2) у (16) систему числення 11011011.1010}
$11011011.1010_2=\frac{1}{2}+\frac{1}{8}+
1+2+0+8+16+0+64+128=219\frac{5}{8}_{10}$

\bigskip
\begin{tabular}{c|c}
	219 & 11 \\
	\hline
	13 & 13 \\
\end{tabular}
\hspace{.1\textwidth}
$219_{10}=DB_{16}$

\medskip
$0.625\cdot16=10=A_{16}$

$11011011.1010_2=DB.A_{16}$

\bigskip
\textbf{Перевірка (за таблицею):}

$$\underbrace{1101}_D\underbrace{1011}_B.\underbrace{1010}_A=DB.A$$

\noindent\textbf{18. Переведіть з (10) у (2) систему числення 164.70}
\begin{multicols}{3}

\begin{tabular}{c|c}
	164 & 0 \\
	\hline
	82 & 0 \\
	\hline
	41 & 1 \\
	\hline
	20 & 0 \\
	\hline
	10 & 0 \\
	\hline
	5 & 1 \\
	\hline
	2 & 0 \\
	\hline
	1 & 1 \\
\end{tabular}

$0.70\cdot2=\underline{1}.4$

$0.4\cdot2=\underline{0}.8$

$0.8\cdot2=\underline{1}.6$

$0.6\cdot2=\underline{1}.2$

$0.2\cdot2=\underline{0}.4$

\end{multicols}
Визначивши період дробової частини,
можемо записати остаточний результат:

$$164.70_{10}=10100100.1(0110)_2$$

\bigskip
\textbf{Перевірка:}

$10100100.1(0110)_2\approx\frac{1}{16}+\frac{1}{8}+\frac{1}{2}
+2^2+2^5+2^7\approx{164.6875}$

\newpage
\noindent\textbf{19.  Переведіть з (10) у (8) систему числення 165.71}
\begin{multicols}{3}
\begin{tabular}{c|c}
	165 & 5 \\
	\hline
	20 & 4 \\
	\hline
	2 & 2 \\
\end{tabular}

\bigskip
$165.71_{10}=245.$

$0.71\cdot8=\underline{5}.68$

$0.68\cdot8=\underline{5}.44$

$0.44\cdot8=\underline{3}.52$

$0.52\cdot8=\underline{4}.16$

$0.16\cdot8=\underline{1}.28$

$0.28\cdot8=\underline{2}.24$

$0.24\cdot8=\underline{1}.92$

$0.92\cdot8=\underline{7}.36$

$0.36\cdot8=\underline{2}.88$

$0.88\cdot8=\underline{7}.04$

$0.04\cdot8=\underline{0}.32$

$0.32\cdot8=\underline{2}.56$

$0.56\cdot8=\underline{4}.48$

$0.48\cdot8=\underline{3}.84$

$0.84\cdot8=\underline{6}.72$

$0.72\cdot8=\underline{5}.76$

$0.76\cdot8=\underline{6}.08$

$0.08\cdot8=\underline{0}.64$

$0.64\cdot8=\underline{5}.12$

$0.12\cdot8=\underline{0}.96$

$0.96\cdot8=\underline{7}.68$

$0.68\cdot8=\underline{5}.44$

\end{multicols}

%Період дробу~--- $553412172702436560507$, отже, число:
Період дробової частини~--- $53412172702436560507$, отже, число:

$$245.5(53412172702436560507)$$

\subsubsection{20.  Переведіть з (10) у  (16)  систему числення 166.72}

\begin{multicols}{2}
	\begin{tabular}{c|c}
		166 & 6 \\
		\hline
		10 & 10 \\
	\end{tabular}

	\bigskip
	Ціла частина~---$A6.$ Перейдемо
	до розрахунку дробової:

	\bigskip

$0.72\cdot16=\underline{11}.52$

$0.52\cdot16=\underline{8}.32$

$0.32\cdot16=\underline{5}.12$

$0.12\cdot16=\underline{1}.92$

$0.92\cdot16=\underline{14}.72$

$0.72\cdot16=\underline{11}.52$

\end{multicols}
$$166.72_{10}=A6.(B851E)_{16}$$

\bigskip
\textbf{Перевірка:}

$A6.(B851E)_{16}\approx6 \cdot 16^0+10 \cdot 16+11 \cdot 16^{-1}+8 \cdot 16^{-2}+5 \cdot 16^{-3}+16^{-4}+14 \cdot 16^{-5}
%\approx\\
\approx 166.71999931335449218750$
\end{document}
