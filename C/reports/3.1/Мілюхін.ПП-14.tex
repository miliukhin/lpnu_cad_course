%xelatex
\documentclass{article}

\usepackage{listings}
\usepackage[T2A]{fontenc}
\usepackage[english,ukrainian]{babel}
\usepackage{fontspec}
\setmainfont{Nimbus Roman}
\usepackage{graphicx}
\usepackage[a4paper,margin=0.5in]{geometry}
\usepackage{tikz}
\usetikzlibrary{calc,positioning,shapes.geometric,shapes.symbols,shapes.misc}

\lstset{
  basicstyle=\ttfamily,
  columns=fullflexible,
  breaklines=true,
  postbreak=\raisebox{0ex}[0ex][0ex]{\color{red}$\hookrightarrow$\space}
}

\tikzset{
    start-end/.style={
        draw,
        rounded rectangle,
%        rounded corners,
    },
    input/.style={ % requires library shapes.geometric
        draw,
        trapezium,
        trapezium left angle=60,
        trapezium right angle=120,
    },
    operation/.style={
        draw,
        rectangle
    },
    loop/.style={ % requires library shapes.misc
        draw,
        chamfered rectangle,
        chamfered rectangle xsep=2cm
    },
    decision/.style={ % requires library shapes.geometric
        draw,
        diamond,
        aspect=#1
    },
    decision/.default=1,
    print/.style={ % requires library shapes.symbols
        draw,
        tape,
        tape bend top=none
    },
}

\begin{document}

\pagestyle{empty}
\begin{center}

	{\fontsize{14}{24}\selectfont МІНІСТЕРСТВО ОСВІТИ І НАУКИ УКРАЇНИ

	НАЦІОНАЛЬНИЙ УНІВЕРСИТЕТ «ЛЬВІВСЬКА ПОЛІТЕХНІКА»

	Інститут комп'ютерних наук та інформаційних технологій

	}

	\vspace{90.4pt} %120.4+16.1
	\begin{figure}[h]
		\centering
		\includegraphics[width=6.5cm,keepaspectratio]{../../../lpnu.png}
	\end{figure}

	{\fontsize{18}{29}\selectfont{Звіт}

	{до лабораторної роботи № 3}

	{на тему:}

	{``Програмування алгоритмів циклічної структури''}

	{З дисципліни}

	{``Алгоритмізація та програмування, частина 1''}

	}
\end{center}

\vspace{12.1pt} %30.1pt
	{\fontsize{14}{22.4}\selectfont
\begin{flushright}
	\textit{Виконав:}

	\textit{студент групи ПП-14}

	\textit{Мілюхін Олександр}

	\textit{Прийняв:}

	\textit{професор Василів К. М.}
\end{flushright}
\vspace{37.4pt} %100.4
\begin{center}
\textit{Львів-2022}
\vspace{37.4pt} %100.4
\end{center}
	}
%{\fontsize{18}{20.7}\selectfont \textbf{Лабораторна робота №1}}
{\fontsize{14}{16.1}\selectfont
\textbf{Мета:}

Ознайомитися з директивами препроцесора мови C, з операторами циклу і функціями вводу-виводу.

\section{Завдання}

Скласти програму для обчислення значень функції для різних значень
аргументу (протабулювати функцію) на вказаному відрізку, використовуючи
три оператори циклу. Обчислити означений інтеграл функції на вказаному
відрізку. Значення аргументу розглянути у вказаній кількості точок, задавши її
як константу препроцесора. Вивести на екран по стовпчиках номер за порядком,
значення аргументу і значення функції, використовуючи можливості форматованого
виводу. Окремо вивести значення обчисленого інтегралу.

\vspace{.5cm}

\renewcommand{\arraystretch}{1.3}
\begin{tabular}{|c| l| c| c|}
	\hline
	№ & Функція & Відрізок & К-сть вузлів \\
	\hline
	5 & $y=arctgx+arctg\frac{1-x}{1+x}$ & [-1.0,1.0] & N = 11 \\
	\hline
\end{tabular}

\bigskip

Схема алгоритму зображена на рисунку 1.

\section{Завдання}
Скласти програму для наближеного обчислення значення функції Y(x) у точці

0 < |х| < 1 за допомогою розкладу в ряд Тейлора S(x). Знайти наближене
значення функції з похибкою менше ε < 0,0001. Значення x та ε вводити з
клавіатури. Вивести на екран точне значення Y(x), знайдене наближене значення S(x) та отриману похибку |S(x)-Y(x)|.

\vspace{.5cm}
\renewcommand{\arraystretch}{2}
\begin{tabular}{|c| l| l|}
	\hline
	№ & Функція & Розклад в ряд Тейлора\\
	\hline
	5 & $Y(x)=e^{cosx}cos(sin(x))$ & $S(x)=\displaystyle\sum_{k=0}^{\infty} \frac{cos(kx)}{k!}$ \\
	\hline
\end{tabular}

\bigskip

Схема алгоритму зображена на рисунку 2.


	\newpage

	%<>
}

\begin{tikzpicture}
\node[start-end] (start) {Початок};
	\node[below of=start,input](outwh){вивід тексту while:};
	\node[below of=outwh,decision=3.9](whiledec){$x <=1.0$};
	\node[below of=whiledec,operation](while){$y = \arctg{x} + \arctg{f}$};
	\node[below of=while,input](reswh){№, x, y};
	\node[below of=reswh,operation](whilee){$x=x+0.1, $№$ = $№$+1$};

	\node[below of=whilee,input](outdo){do while:};
	\node[below of=outdo,operation](do){$y = \arctg{x} + \arctg{f}$};
	\node[below of=do,input](resdo){№, x, y};
	\node[below of=resdo,operation](doo){$x=x+0.1, $№$ = $№$+1$};
	\node[below of=doo,decision=3.9](dodec){$x <=1.0$};

	\node[below of=dodec,operation](afterdo){№=1};

	\node[below of=afterdo,input](outfor){for:};
	\node[below of=outfor,loop](fordec){$x=-1;x<=1;x=x+0.1$};
	\node[below of=fordec,operation](for){$y = \arctg{x} + \arctg{f}$};
	\node[below of=for,input](resfor){№, x, y};
	%\node[below of=reswh,operation](whilee){$x=x+0.1, $№$ = $№$+1$};

	\draw [thin] (start) -- (outwh);
	\draw [thin] (outwh) -- (whiledec);
	\draw [thin] (whiledec) -- (while);
	\draw [thin] (while) -- (reswh);
	\draw [thin] (whilee) -- (reswh);
	\draw [-latex] (whilee) -- (-3,-5) -- (-3,-1.4) -- (0,-1.4);
	\draw [-latex] (whiledec) -- (2.5,-2) -- (2.5,-5.5) -- (0,-5.5) -- (outdo);

	\draw [thin] (outdo) -- (do);
	\draw [thin] (do) -- (resdo);
	\draw [thin] (resdo) -- (doo);
	\draw [thin] (doo) -- (dodec);
	\draw [thin] (dodec) -- node[anchor=east] {ні} (afterdo);
	\draw [-latex] (dodec) -- node[anchor=south] {так} (-3,-10) -- (-3,-6.5) -- (0,-6.5);
	\draw [thin] (afterdo) -- (outfor);

	\draw [thin] (outfor) -- (fordec);
	\draw [thin] (fordec) -- (for);

\node[below of=resfor,loop=3.9] (dec) {$-1<=x<=1$};

	\draw [-latex] (fordec) -- (3,-13) -- (3,-15.5) -- (0,-15.5) -- (dec);
	\draw [thin] (for) -- (resfor);
	\draw [-latex] (resfor) -- (-3,-15) -- (-3,-13) -- (fordec);

	%integral

%\node[below=5mm of op1,loop=3.9] (dec) {$-1<=x<=1$};
\node[below=5mm of dec,operation] (op2) {$S_i = x_i + \frac{1-(-1),}{11} \cdot f(x_i)$; $S = S + S_i$};
\node[left=5mm of op2] (ye) {};
\node[right=5mm of op2] (no) {ні};
\node[below=10mm of op2,input] (outint) {S};
\node[below=5mm of op2] (space) {};
\node[below of=outint,start-end] (end) {Кінець};
\node[below of=end,] (rys1) {рис. 1};

%	\draw [thin] (op1) -- (dec);
	\draw [thin] (dec) -- node[anchor=east] {так} (op2); % |-
	\draw [-latex] (op2) -- (-4,-17.2) -- (-4,-16) -- (dec); % |-
	\draw [-latex] (no) |- (0,-18) -- (outint); % |-
	\draw [thin] (no) |- (dec); % |-
	\draw [thin] (outint) -- (end); % |-

	%taylor

	\node[right=5cm of start,start-end] (start2) {Початок};
	\node[below of=start2,input] (inp2) {x, похибка};
	\node[below of=inp2,operation] (op02) {обчислення $Y(x)$};
	\node[below of=op02,loop] (tayloop) {$|Y(x)-S(x)|>\epsilon; k=k+1$};
	\node[below of=tayloop,operation] (tayloopbd) {$S(x) = \Sigma^{\infty}_{k=0}\frac{\cos(kx)}{k!}$;  $|Y(x)-S(x)|$ };
	\node[below of=tayloopbd,input] (outtaylor) {$Y(x), S(x)$, похибка};
	\node[below of=outtaylor,start-end] (endtaylor) {Кінець};
	\node[below=10mm of endtaylor] (rys.2) {рис. 2};

	\draw [thin] (start2) -- (inp2);
	\draw [thin] (inp2) -- (op02);
	\draw [thin] (op02) -- (tayloop);
	\draw [-latex] (tayloop) -- (10,-3) -- (10,-4.5) -- (6.68,-4.5) -- (outtaylor);
	\draw [thin] (tayloop) -- (tayloopbd);
	\draw [-latex] (tayloopbd) -- (3,-4) -- (3,-3) -- (tayloop);
	\draw [thin] (outtaylor) -- (endtaylor);

\end{tikzpicture}

\section{Програмний код до першого завдання}
\begin{lstlisting}[frame=single]

#include <stdio.h>
#include <math.h>

#define N 11

int integral(){

	double y=0.0, x=-1.0, f=0.0, diff=2.0/N, integral = 0.0;

	int i = 1;

	f=(1-x)/(1+x);

	while(x<=1.){
		y = atan (x) + atan (f);
		f=(1-x)/(1+x);
		integral = integral + diff * y;
		printf("integr = %lf\tx = %lf\ty = %lf\ti = %d\n",integral,x,y,i);
		i++;
		f=(1-x)/(1+x);
		x=x+diff;
	}
	printf("Integral on segment [-1.0,1.0], calculated using left rectangle method with step11: %lf\n",integral);

	return 0;
}

	double y=0.0, x=-1.0, f=0.0;

int main(){

	printf("while:\n");
	int num = 1;
	f=(1-x)/(1+x);
	while(x<=1.0){
		y = atan(x)+atan(f);
		f=(1-x)/(1+x);
		printf("%d.\t%lf\t%lf\n",num,x,y);
		x=x+.1;
		num++;
		}

	x=-1.0; num = 1;
	f=(1-x)/(1+x);
	printf("do while:\n");
	do{
		y = atan(x)+atan(f);
		f=(1-x)/(1+x);
		printf("%d.\t%lf\t%lf\n",num,x,y);
		x=x+.1;
		num++;
		}
	while(x<=1.0);

	num = 1;
	printf("for:\n");
	x=-1.0; f=(1-x)/(1+x);

	for(x=-1.0;x<=1.0;x+=.1){
		y = atan(x)+atan(f);
		f=(1-x)/(1+x);
		printf("%d.\t%lf\t%lf\n",num,x,y);
		num++;
	}

	integral();

	return 0;
}


\end{lstlisting}

{\fontsize{14}{16.1}\selectfont
Вивід:}

\begin{lstlisting}

while:
1.      -1.000000       0.785398
2.      -0.900000       0.837981
3.      -0.800000       0.843472
4.      -0.700000       0.849413
5.      -0.600000       0.855705
6.      -0.500000       0.862170
7.      -0.400000       0.868539
8.      -0.300000       0.874448
9.      -0.200000       0.879459
10.     -0.100000       0.883125
11.     -0.000000       0.885067
12.     0.100000        0.885067
13.     0.200000        0.883125
14.     0.300000        0.879459
15.     0.400000        0.874448
16.     0.500000        0.868539
17.     0.600000        0.862170
18.     0.700000        0.855705
19.     0.800000        0.849413
20.     0.900000        0.843472
21.     1.000000        0.837981
do while:
1.      -1.000000       0.785398
2.      -0.900000       0.837981
3.      -0.800000       0.843472
4.      -0.700000       0.849413
5.      -0.600000       0.855705
6.      -0.500000       0.862170
7.      -0.400000       0.868539
8.      -0.300000       0.874448
9.      -0.200000       0.879459
10.     -0.100000       0.883125
11.     -0.000000       0.885067
12.     0.100000        0.885067
13.     0.200000        0.883125
14.     0.300000        0.879459
15.     0.400000        0.874448
16.     0.500000        0.868539
17.     0.600000        0.862170
18.     0.700000        0.855705
19.     0.800000        0.849413
20.     0.900000        0.843472
21.     1.000000        0.837981
for:
1.      -1.000000       0.785398
2.      -0.900000       0.837981
3.      -0.800000       0.843472
4.      -0.700000       0.849413
5.      -0.600000       0.855705
6.      -0.500000       0.862170
7.      -0.400000       0.868539
8.      -0.300000       0.874448
9.      -0.200000       0.879459
10.     -0.100000       0.883125
11.     -0.000000       0.885067
12.     0.100000        0.885067
13.     0.200000        0.883125
14.     0.300000        0.879459
15.     0.400000        0.874448
16.     0.500000        0.868539
17.     0.600000        0.862170
18.     0.700000        0.855705
19.     0.800000        0.849413
20.     0.900000        0.843472
21.     1.000000        0.837981
Integral on segment [-1.0,1.0], calculated using left rectangle method with step11: 1.838274
\end{lstlisting}

\section{Програмний код до другого завдання}

\begin{lstlisting}[frame=single]
#include <stdio.h>
#include <math.h>

double y_x, s_x = 1, x = 0.1, eps = 0.0001, pox, dil;

int main(){

	printf("Enter x value (By default - 0.1):\n");
	scanf("%lf", &x);
	printf("Enter the error value (By default - 0.0001):\n");
	scanf("%lf", &eps);

	y_x = exp(cos(x))*cos(sin(x));

	double kfact = 1, chys;
	int k = 1;

	for(pox = fabs(y_x-s_x);pox>eps;k++){
		kfact *=k;
		chys = cos(k*x);
		dil = (chys/kfact);
		s_x += dil;
		pox = fabs(y_x-s_x);
	}

	printf("Y(x) = %lf\n", y_x);
	printf("S(x) = %lf\n", s_x);
	printf("Error: %lf\n", pox);

}
\end{lstlisting}

{\fontsize{14}{16.1}\selectfont
Вивід:}

\begin{lstlisting}
Enter x value (By default - 0.1):
1
Enter the error value (By default - 0.0001):
2
Y(x) = 1.143836
S(x) = 1.000000
Error: 0.143836
\end{lstlisting}

{\fontsize{14}{16.1}\selectfont
\textbf{Висновок:}

У процесі виконання роботи я покращив свої вміння працювати з циклами та програмувати математичні задачі мовою C (зокрема обчислення табулювання функцій, обчислення означеного інтегралу та знаходження наближеного значення функції за допомогою розкладу в ряд Тейлора).

}

\end{document}
