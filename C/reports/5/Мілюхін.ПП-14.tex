%xelatex
\documentclass[12pt]{extreport}
%\documentclass[14pt]{extreport}

\usepackage{listings}
%\usepackage{verbatim}
\usepackage[T2A]{fontenc}
\usepackage[english,ukrainian]{babel}
\usepackage{mathspec}
\setallmainfonts{Nimbus Roman}
\usepackage{graphicx}
\usepackage[a4paper,margin=0.5in]{geometry}
\usepackage{tikz}
\usetikzlibrary{calc,positioning,shapes.geometric,shapes.symbols,shapes.misc}
\usepackage{indentfirst}
\usepackage{moreverb}
\usepackage{subfiles}

\lstset{
  basicstyle=\ttfamily,
  columns=fullflexible,
  breaklines=true,
  postbreak=\raisebox{0ex}[0ex][0ex]{\color{red}$\hookrightarrow$\space}
}

\begin{document}

\subfile{title.tex}

%{\fontsize{18}{20.7}\selectfont \textbf{Лабораторна робота №1}}
%{\fontsize{14}{16.1}\selectfont
\subsubsection*{Мета роботи}
Навчитися використовувати символьні масиви для розв’язання задач
роботи зі стрічками.
\bigskip


\subsubsection*{Завдання № 1}
Ввести з клавіатури своє прізвище, ім'я та по батькові як одне текстове
дане. Вивести ім'я та кількість букв у прізвищі.

\bigskip
Схема алгоритму зображена на рис. 1.

\begin{figure}[h]
	\centering
	\subfile{fch.tex}
	\caption{Схема алгоритму для першого завдання}
\end{figure}

\subsubsection*{Програмний код до першого завдання}

\begin{lstlisting}[frame=single]
#include <stdio.h>
#include <string.h>

char fns[128];
int n;

int main(){

	printf(">");
	fgets(fns, 128, stdin);
	char* fn = strtok(fns, " ");
	char* mn = strtok(NULL, " ");
	char* ln = strtok(NULL, " ");
	printf("%s\n", fn);
	printf("%d\n", strlen(ln)-1);

	return 0;
}
\end{lstlisting}

Вивід:

\begin{lstlisting}
>Oleksandr Pavlovych Miliukhin
Oleksandr
9
\end{lstlisting}

\subsubsection*{Завдання № 2}

Символьні рядки S1 та S2 довжиною до 100 символів вводити з клавіатури.
Змінити порядок символів у S1 на зворотній. Замінити усі маленькі
приголосні S2 на великі.

\bigskip
Схема алгоритму зображена на рис. \ref{task2}

\begin{figure}[h]
	\centering
	\subfile{fch2.tex}
%	\includegraphics[width=.7\textwidth]{~/pic-selected-221205-1221-56.png}
	\caption{Схема алгоритму для другого завдання;
	sln~--- довжина введеної стрічки}
	\label{task2}
\end{figure}

\subsubsection*{Програмний код до другого завдання}

\begin{lstlisting}[frame=single]
#include <stdio.h>
#include <string.h>
#include <ctype.h>

char s1[100], s2[100], s1rev[100];

int main()
{
	printf(">");
	fgets(s1, 100, stdin);

	for(int i=0;i<=strlen(s1);i++){
            	s1rev[strlen(s1)-1-i] = s1[i-1];
	}
		s1rev[strlen(s1)-1]=s1[strlen(s1)-1]; // nepeHeceHHR "\0"

	fputs(s1rev,stdout);

	printf(">");
	fgets(s2, 100, stdin);
	int i=0;
	while(i<strlen(s1)) {
      		putchar (toupper(s2[i]));
      		i++;
   	}
	return 0;
}
\end{lstlisting}

Вивід:

\begin{lstlisting}
>some text
txet emos
>some more text
SOME MORE TEXT
\end{lstlisting}

\subsubsection*{Висновок:}
Виконанням цієї лабораторної роботи я закріпив свої знання
про роботу з символьними рядками у мові C.

\end{document}
