\documentclass[a4paper,14pt]{extreport}
\usepackage{tikz}
\usepackage[english,ukrainian]{babel}
\usepackage[T2A]{fontenc}
\usetikzlibrary{calc,positioning,shapes.geometric,shapes.symbols,shapes.misc}
\pagestyle{empty}
\begin{document}
\tikzset{
    start-end/.style={
        draw,
        rounded rectangle,
%        rounded corners,
    },
    input/.style={ % requires library shapes.geometric
        draw,
        trapezium,
        trapezium left angle=60,
        trapezium right angle=120,
    },
    operation/.style={
        draw,
        rectangle
    },
    loop/.style={ % requires library shapes.misc
        draw,
        chamfered rectangle,
        chamfered rectangle xsep=2cm
    },
    decision/.style={ % requires library shapes.geometric
        draw,
        diamond,
    text width=6cm,
        aspect=#1
    },
    decision/.default=1,
    print/.style={ % requires library shapes.symbols
        draw,
        tape,
        tape bend top=none
    },
}


\begin{tikzpicture}
\node[start-end] (start) {Початок};
\node[below of=start,input](inp){Ввід};
\node[below of=inp,operation](op1){Розбиття символьного рядка за символами пропуску};
\node[below of=op1,input](ou1){Вивід першого слова};
\node[below of=ou1,input](ou2){Вивід кількості букв у другому слові};
%\node[below=5mm of inp,loop](gen){i = 0; i<N; i = i+1};
\node[below of=ou2,start-end] (end) {Кінець};

	\draw [thin] (start) -- (inp);
	\draw [thin] (inp) -- (op1);
	\draw [thin] (ou1) -- (op1);
	\draw [thin] (ou1) -- (ou2);
	\draw [thin] (end) -- (ou2);
%	\draw [thin] (countop2) -- (countdec2);

\end{tikzpicture}

\end{document}
