%xelatex
\documentclass[14pt]{extreport}
%\documentclass[14pt]{extreport}

\usepackage{listings}
%\usepackage{verbatim}
\usepackage[T2A]{fontenc}
\usepackage[english,ukrainian]{babel}
\usepackage{mathspec}
\setallmainfonts{Nimbus Roman}
\usepackage{graphicx}
\usepackage[a4paper,margin=0.5in]{geometry}
\usepackage{tikz}
\usetikzlibrary{calc,positioning,shapes.geometric,shapes.symbols,shapes.misc}
\usepackage{indentfirst}
\usepackage{moreverb}
\usepackage{subfiles}

\lstset{
  basicstyle=\ttfamily,
  columns=fullflexible,
  breaklines=true,
  postbreak=\raisebox{0ex}[0ex][0ex]{\color{red}$\hookrightarrow$\space}
}

\tikzset{
    start-end/.style={
        draw,
        rounded rectangle,
%        rounded corners,
    },
    input/.style={ % requires library shapes.geometric
        draw,
        trapezium,
        trapezium left angle=60,
        trapezium right angle=120,
    },
    operation/.style={
        draw,
        rectangle
    },
    loop/.style={ % requires library shapes.misc
        draw,
        chamfered rectangle,
        chamfered rectangle xsep=2cm
    },
    decision/.style={ % requires library shapes.geometric
        draw,
        diamond,
        aspect=#1
    },
    decision/.default=1,
    print/.style={ % requires library shapes.symbols
        draw,
        tape,
        tape bend top=none
    },
}

\begin{document}

\pagestyle{empty}
\begin{center}

	{\fontsize{14}{24}\selectfont МІНІСТЕРСТВО ОСВІТИ І НАУКИ УКРАЇНИ

	НАЦІОНАЛЬНИЙ УНІВЕРСИТЕТ «ЛЬВІВСЬКА ПОЛІТЕХНІКА»

	Інститут комп'ютерних наук та інформаційних технологій

	}

	\vspace{90.4pt} %120.4+16.1
	\begin{figure}[h]
		\centering
		\includegraphics[width=6.5cm,keepaspectratio]{../../../lpnu.png}
	\end{figure}

	{\fontsize{18}{29}\selectfont{Звіт}

	{до лабораторної роботи № 4}

	{З дисципліни}

	{``Алгоритмізація та програмування, частина 1''}

	{на тему:}

	{``Масиви і файли в мові програмування C''}

	}
\end{center}

\vspace{12.1pt} %30.1pt
	{\fontsize{14}{22.4}\selectfont
\begin{flushright}
	\textit{Виконав:}

	\textit{студент групи ПП-14}

	\textit{Мілюхін Олександр}

	\textit{Прийняв:}

	\textit{професор Василів К. М.}
\end{flushright}
\vspace{37.4pt} %100.4
\begin{center}
\textit{Львів-2022}
\vspace{37.4pt} %100.4
\end{center}
	}
%{\fontsize{18}{20.7}\selectfont \textbf{Лабораторна робота №1}}
%{\fontsize{14}{16.1}\selectfont
\subsubsection*{Мета роботи:}

Навчитися використовувати масиви та файли при розв’язанні задач.

\bigskip


\subsubsection*{Завдання № 1}
Дано натуральне число N (задати довільно, як константу препроцесора)
і одновимірний масив A0, A1, …, AN-1 цілих чисел (згенерувати додатні
та від’ємні
елементи випадковим чином, за допомогою функції бібліотеки <stdlib.h>
rand()). Визначити число пар двох однакових додатних чисел, наприклад, чотири
числа утворюють дві пари.

Схема алгоритму зображена на рис. 1.

\begin{figure}[h]
	\centering
	\subfile{fch.tex}
	\caption{Схема алгоритму для першого завдання}
\end{figure}

\subsubsection*{Програмний код до першого завдання}

\begin{lstlisting}[frame=single]
#include <stdio.h>
#include <stdlib.h>
#include <time.h>
#include <math.h>
#define N 4

int m [N]={1,2,2,3};

int main(){

	srand(time(NULL));

	printf("Array items:\n");

	for(int i = 0; i<N; i++){
		int n = rand() % 9;
		if (n > 4)
			m[i] = -rand() % 10;
		else
			m[i] = rand() % 10;

			printf("%d\t", m[i] );
	}
			printf("\n\n");

	int count, paircount;

	for(int i = 0; i<N; i++){
	if(fmod(m[i],2) == 0){
		count++;
		if(fmod(count,2) == 0)
			paircount++;
		}
	}
			printf("even number pairs :\t%d\n", paircount);


}

\end{lstlisting}

Вивід:
\begin{lstlisting}
Array items:
5       6       -3      -9

even number pairs :     0
\end{lstlisting}

\subsubsection*{Завдання № 2}

У файл F1.txt попередньо записати матрицю цілих чисел А(N,N)
(згенерувати випадковим чином, N задати довільно, як константу
препроцесора). Прочитати матрицю з файлу, виконати описані нижче дії, їх
результати записати в файл F2.txt.

Схема алгоритму зображена на рис. \ref{task2}

\begin{figure}[h]
	\centering
	\subfile{fch2.tex}
%	\includegraphics[width=.7\textwidth]{~/pic-selected-221205-1221-56.png}
	\caption{Схема алгоритму для другого завдання}
	\label{task2}
\end{figure}

\subsubsection*{Програмний код до другого завдання}

\begin{lstlisting}[frame=single]
#include <stdio.h>
#include <stdlib.h>
#include <time.h>
#include <math.h>
#define N 4

int m [N][N], m2 [N][N], i, j;

int gen(){
	srand(time(NULL));

	printf("Array items:\n");

	FILE *fp = fopen("F1.txt", "w");

	for(i = 0; i<N; i++){
		for(j = 0; j<N; j++){
			int n = rand() % 9;
			if (n > 4)
				m[i][j] = -rand() % 10;
			else
				m[i][j] = rand() % 10;
			fprintf(fp, "%d\t", m[i][j]);
			printf("%d\t", m[i][j]);
		}
			fprintf(fp, "\n");
			printf("\n");

	}
			printf("\n");

	fclose(fp);
	}

int transpose(){
	printf("\n");

	FILE *f1 = fopen("F1.txt", "r");

	for(i = 0; i<N; i++){
		for(j = 0; j<N; j++){
			fscanf(f1, "%d\t", &m[i][j] );
			printf("%d\t", m[i][j] );
		}
			printf("\n");
	}

	fclose(f1);

	printf("Transposed:\n");

	FILE *f2 = fopen("F2.txt", "w");

	for(i = 0; i<N; i++){
		for(j = 0; j<N; j++){
			m2[i][j]=m[j][i];
			fprintf(f2, "%d\t", m2[i][j] );
			printf("%d\t", m2[i][j] );
		}
			fprintf(f2, "\n");
			printf("\n");
	}

	fclose(f2);
}

int main(){
	gen();
	transpose();
	return 0;
}

\end{lstlisting}

Вивід:
\begin{lstlisting}
The matrix:
-1      -4      6       -7
-1      -7      4       7
2       -6      -9      9
-2      2       9       -8

Transposed:
-1      -1      2       -2
-4      -7      -6      2
6       4       -9      9
-7      7       9       -8
\end{lstlisting}

\subsubsection*{Висновок:}

Виконавши цю лабораторну роботу, я
застосував на практиці та закріпив
здобуті знання про роботу
з масивами у мові C.



\end{document}
