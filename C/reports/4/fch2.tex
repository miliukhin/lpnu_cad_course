\documentclass[a4paper,14pt]{extreport}
\usepackage{tikz}
\usepackage[english,ukrainian]{babel}
\usepackage[T2A]{fontenc}
\usetikzlibrary{calc,positioning,shapes.geometric,shapes.symbols,shapes.misc}
\pagestyle{empty}
\tikzset{
    start-end/.style={
        draw,
        rounded rectangle,
%        rounded corners,
    },
    input/.style={ % requires library shapes.geometric
        draw,
        trapezium,
        trapezium left angle=60,
        trapezium right angle=120,
    },
    operation/.style={
        draw,
        rectangle
    },
    loop/.style={ % requires library shapes.misc
        draw,
        chamfered rectangle,
        chamfered rectangle xsep=2cm
    },
    decision/.style={ % requires library shapes.geometric
        draw,
        diamond,
    text width=6cm,
        aspect=#1
    },
    decision/.default=1,
    print/.style={ % requires library shapes.symbols
        draw,
        tape,
        tape bend top=none
    },
}

\begin{document}

\begin{tikzpicture}
\node[start-end] (start) {Початок};
%\node[below=5mm of start,input](inp){Ввід};
%\node[below of=inp,operation] (op0) {};
	\node[below=5mm of start,loop](gen){i = 0; i<N; i = i+1};
	\node[below=5mm of gen,loop](genn){j = 0; j<N; j = j+1};
	\node[below=5mm of genn,operation](genops){Генерація ij-го елемента масиву};
	\node[below=5mm of genops,input](genout){ел$_{ij}$ в термінал та у файл F1.txt};
	\node[below=2mm of genout,input](genl){символ переходу рядка};
%	\node[below=10mm of genout,loop](count){i = 0; i<N; i = i+1};

	\node[below=5mm of genl,loop](trans){i = 0; i<N; i = i+1};
	\node[below=5mm of trans,loop](transs){j = 0; j<N; j = j+1};
	\node[below=5mm of transs,input](transin){ввід ел$_{ji}$ з файлу F1.txt};
	\node[below=3mm of transin,input](transnl){символ переходу рядка};

	\node[below=5mm of transnl,loop](transp){i = 0; i<N; i = i+1};
	\node[below=5mm of transp,loop](transpp){j = 0; j<N; j = j+1};
	\node[below=5mm of transpp,operation](transpop)
	{ел$_{ij}$ = ел$_{ji}$};
	\node[below=5mm of transpop,input](transpout)
	{запис елемента у файл F2.txt};
	\node[below=3mm of transpout,input](transpnl){символ переходу рядка};
%	\node[below=5mm of transs,operation](transops){Зчитування ij-го елемента масиву};

	\node[below=5mm of transpnl,start-end] (end) {Кінець};
	%\node[below=5mm of countdec,input](countopr){Кількість пар чисел, що
	%діляться на 2};
	%\node[below=5mm of countdec,input](outcount){Кількість пар чисел, що

	\draw [thin] (start) -- (gen);
	\draw [-latex] (gen) -- (5,-1.4) -- (5,-7.3) -- (0,-7.3) -- (trans);
	\draw [thin] (genn) -- (genops);
	\draw [thin] (gen) -- (genn);
	\draw [-latex] (genn) -- (4.5,-3) -- (4.5,-6.69)
	-- (genl) -- (-4.5,-6.69)
	-- (-4.5,-1.38) -- (gen);
	\draw [thin] (genops) -- (genout);
	\draw [-latex] (genout) -- (-4,-5.7) -- (-4,-2.95) -- (genn);

	\draw [thin] (trans) -- (transs);
	\draw [-latex] (trans) -- (5,-8.1) -- (5,-12.8)
	-- (0,-12.8) -- (transp);

	\draw [-latex] (transs) -- (4.5,-9.65) -- (4.5,-12.2) --
	(transnl) -- (-4.5,-12.2)
	-- (-4.5,-8.05) -- (trans);
	\draw [-latex] (transin) -- (-4,-11.1) -- (-4,-9.6) -- (transs);
	\draw [thin] (transs) -- (transin);

	\draw [thin] (transp) -- (transpp);
	\draw [thin] (transpp) -- (transpop);
	\draw [thin] (transpop) -- (transpout);

	\draw[-latex] (transp) -- (5,-13.65) -- (5,-19.5)
	-- (0,-19.5) -- (end);

	\draw[-latex] (transpout) -- (-4.5,-17.8) -- (-4.5,-15.2)
	-- (transpp);

	\draw[-latex] (transpp) -- (4.5,-15.2) -- (4.5,-18.9) --
	(transpnl) -- (-5,-18.9) -- (-5,-13.64) -- (transp);

%	\draw [thin] () -- (end);
%	\draw [thin] (6.5,-9.2) -- (6.5,-12);
%	\draw [thin] (countop2) -- (countdec2);

\end{tikzpicture}

\end{document}
