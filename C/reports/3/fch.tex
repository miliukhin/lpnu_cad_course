\documentclass[a4paper]{article}
\usepackage{tikz}
\usepackage[english,ukrainian]{babel}
\usepackage[T2A]{fontenc}
\usetikzlibrary{calc,positioning,shapes.geometric,shapes.symbols,shapes.misc}
\pagestyle{empty}
\tikzset{
    start-end/.style={
        draw,
        rounded rectangle,
%        rounded corners,
    },
    input/.style={ % requires library shapes.geometric
        draw,
        trapezium,
        trapezium left angle=60,
        trapezium right angle=120,
    },
    operation/.style={
        draw,
        rectangle
    },
    loop/.style={ % requires library shapes.misc
        draw,
        chamfered rectangle,
        chamfered rectangle xsep=2cm
    },
    decision/.style={ % requires library shapes.geometric
        draw,
        diamond,
        aspect=#1
    },
    decision/.default=1,
    print/.style={ % requires library shapes.symbols
        draw,
        tape,
        tape bend top=none
    },
}

\begin{document}
\begin{tikzpicture}
\node[start-end] (start) {Початок};
\node[below of=start,input](inp){Ввід};
\node[below of=inp,operation] (op0) {поперед. обчисл. частини ф-ї};
	\node[below of=op0,input](outwh){while:};
	\node[below of=outwh,decision=3.9](whiledec){$x <=1.0$};
	\node[below of=whiledec,operation](while){$y = \arctg{x} + \arctg{f}$};
	\node[below of=while,input](reswh){№, x, y};
	\node[below of=reswh,operation](whilee){$x=x+0.1, $№$ = $№$+1$};

	\node[below of=whilee,input](outdo){do while:};
	\node[below of=outdo,operation](do){$y = \arctg{x} + \arctg{f}$};
	\node[below of=do,input](resdo){№, x, y};
	\node[below of=resdo,operation](doo){$x=x+0.1, $№$ = $№$+1$};
	\node[below of=doo,decision=3.9](dodec){$x <=1.0$};

	\node[below of=dodec,operation](afterdo){№=1};

	\node[below of=afterdo,input](outfor){for:};
	\node[below of=outfor,loop](fordec){$x=-1;x<=1;x=x+0.1$};
	\node[below of=fordec,operation](for){$y = \arctg{x} + \arctg{f}$};
	\node[below of=for,input](resfor){№, x, y};
	%\node[below of=reswh,operation](whilee){$x=x+0.1, $№$ = $№$+1$};

	\draw [thin] (start) -- (inp);
	\draw [thin] (inp) -- (op0);
	\draw [thin] (op0) -- (outwh);
	\draw [thin] (outwh) -- (whiledec);
	\draw [thin] (whiledec) -- (while);
	\draw [-latex] (reswh) -- (-3,-6) -- (-3,-3.4) -- (0,-3.4);
	\draw [thin] (while) -- (reswh);
	\draw [-latex] (whiledec) -- (2.5,-4) -- (2.5,-6.5) -- (0,-6.5) -- (whilee);
	\draw [thin] (whilee) -- (outdo);

	\draw [thin] (outdo) -- (do);
	\draw [thin] (do) -- (resdo);
	\draw [thin] (resdo) -- (doo);
	\draw [thin] (doo) -- (dodec);
	\draw [thin] (dodec) -- node[anchor=east] {ні} (afterdo);
	\draw [-latex] (dodec) -- node[anchor=south] {так} (-3,-12) -- (-3,-8.5) -- (0,-8.5);
	\draw [thin] (afterdo) -- (outfor);

	\draw [thin] (outfor) -- (fordec);
	\draw [thin] (fordec) -- (for);

\node[below of=resfor,operation] (op1) {поперед. обчисл. частини ф-ї};

	\draw [-latex] (fordec) -- (3,-15) -- (3,-17.5) -- (0,-17.5) -- (op1);
	\draw [thin] (for) -- (resfor);
	\draw [-latex] (resfor) -- (-3,-17) -- (-3,-15) -- (fordec);

	%integral

\node[below=5mm of op1,loop=3.9] (dec) {$-1<=x<=1$};
\node[below=5mm of dec,operation] (op2) {$S_i = x_i + \frac{1-(-1),}{11} \cdot f(x_i)$; $S = S + S_i$};
\node[left=5mm of op2] (ye) {};
\node[right=5mm of op2] (no) {ні};
\node[below=10mm of op2,input] (outint) {S};
\node[below=5mm of op2] (space) {};
\node[below of=outint,start-end] (end) {Кінець};
\node[below of=end,] (rys1) {рис. 1};

	\draw [thin] (op1) -- (dec);
	\draw [thin] (dec) -- node[anchor=east] {так} (op2); % |-
	\draw [-latex] (op2) -- (-4,-20.33) -- (-4,-19.1) -- (dec); % |-
	\draw [-latex] (no) |- (0,-21) -- (outint); % |-
	\draw [thin] (no) |- (dec); % |-
	\draw [thin] (outint) -- (end); % |-

	%taylor

	\node[right=5cm of start,start-end] (start2) {Початок};
	\node[below of=start2,input] (inp2) {x, похибка};
	\node[below of=inp2,operation] (op02) {обчислення $Y(x)$};
	\node[below of=op02,loop] (tayloop) {$|Y(x)-S(x)|>\epsilon; k=k+1$};
	\node[below of=tayloop,operation] (tayloopbd) {$S(x) = \Sigma^{\infty}_{k=0}\frac{\cos(kx)}{k!}$;  $|Y(x)-S(x)|$ };
	\node[below of=tayloopbd,input] (outtaylor) {$Y(x), S(x)$, похибка};
	\node[below of=outtaylor,start-end] (endtaylor) {Кінець};
	\node[below=10mm of endtaylor] (rys.2) {рис. 2};

	\draw [thin] (start2) -- (inp2);
	\draw [thin] (inp2) -- (op02);
	\draw [thin] (op02) -- (tayloop);
	\draw [-latex] (tayloop) -- (10,-3) -- (10,-4.5) -- (6.68,-4.5) -- (outtaylor);
	\draw [thin] (tayloop) -- (tayloopbd);
	\draw [-latex] (tayloopbd) -- (3,-4) -- (3,-3) -- (tayloop);
	\draw [thin] (outtaylor) -- (endtaylor);

%\node[below= 5mm of lp,decision] (dec) {Decision};
%\node[right= 10mm of dec,decision=1.6] (dec2) {Decision};
%\node[right= 10mm of dec2,decision=2.5] (dec3) {Decision};
%\node[below= 5mm of dec,print] (pr) {Print};


\end{tikzpicture}
\end{document}
