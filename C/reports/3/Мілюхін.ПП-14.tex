%xelatex
\documentclass{article}

\usepackage{listings}
\usepackage[T2A]{fontenc}
\usepackage[english,ukrainian]{babel}
\usepackage{fontspec}
\setmainfont{Nimbus Roman}
\usepackage{graphicx}
\usepackage[a4paper,margin=0.5in]{geometry}
\usepackage{subfiles}

\usepackage{tikz}
\usetikzlibrary{calc,positioning,shapes.geometric,shapes.symbols,shapes.misc}

\lstset{
  basicstyle=\ttfamily,
  columns=fullflexible,
  breaklines=true,
  postbreak=\raisebox{0ex}[0ex][0ex]{\color{red}$\hookrightarrow$\space}
}

\tikzset{
    start-end/.style={
        draw,
        rounded rectangle,
%        rounded corners,
    },
    input/.style={ % requires library shapes.geometric
        draw,
        trapezium,
        trapezium left angle=60,
        trapezium right angle=120,
    },
    operation/.style={
        draw,
        rectangle
    },
    loop/.style={ % requires library shapes.misc
        draw,
        chamfered rectangle,
        chamfered rectangle xsep=2cm
    },
    decision/.style={ % requires library shapes.geometric
        draw,
        diamond,
        aspect=#1
    },
    decision/.default=1,
    print/.style={ % requires library shapes.symbols
        draw,
        tape,
        tape bend top=none
    },
}

\begin{document}

\pagestyle{empty}
\begin{center}

	{\fontsize{14}{24}\selectfont МІНІСТЕРСТВО ОСВІТИ І НАУКИ УКРАЇНИ

	НАЦІОНАЛЬНИЙ УНІВЕРСИТЕТ «ЛЬВІВСЬКА ПОЛІТЕХНІКА»

	Інститут комп'ютерних наук та інформаційних технологій

	}

	\vspace{90.4pt} %120.4+16.1
	\begin{figure}[h]
		\centering
		\includegraphics[width=6.5cm,keepaspectratio]{../../../lpnu.png}
	\end{figure}

	{\fontsize{18}{29}\selectfont{Звіт}

	{до лабораторної роботи № 3}

	{на тему:}

	{``Програмування алгоритмів розгалуженої структури''}

	{З дисципліни}

	{``Алгоритмізація та програмування, частина 1''}

	}
\end{center}

\vspace{12.1pt} %30.1pt
	{\fontsize{14}{22.4}\selectfont
\begin{flushright}
	\textit{Виконав:}

	\textit{студент групи ПП-14}

	\textit{Мілюхін Олександр}

	\textit{Прийняв:}

	\textit{професор Василів К. М.}
\end{flushright}
\vspace{37.4pt} %100.4
\begin{center}
\textit{Львів-2022}
\vspace{37.4pt} %100.4
\end{center}
	}
%{\fontsize{18}{20.7}\selectfont \textbf{Лабораторна робота №1}}
{\fontsize{14}{16.1}\selectfont
\textbf{Мета:}

Розробити схему алгоритму і програмний код для обчислення функцій:

\begin{enumerate}
	\item $\alpha = \ln(y^{-\sqrt{|x|}}) \cdot (x-y/2)+sin^2\arctg(z)$
	\item $y = x - \lg(x+2.5) + 3\cdot(e^x-e^{-x})$
\end{enumerate}

У схемі алгоритму та програмному коді врахувати ситуації з можливими
невизначеностями, які можуть бути спричинені певними значеннями вхідних даних
або проміжних результатів обчислення.
Вхідні дані ввести з консолі, вихідні - вивести на консоль.
Для програмування розгалужень застосувати оператор умовного переходу if та
тринарний оператор.

\bigskip

Схема алгоритму зображена на рисунку 1:

}

\subfile{fch.tex}

\section{Програмний код до першого завдання}

\begin{lstlisting}[frame=single]

#include <stdio.h>
#include <math.h>

double x, y, z, i, r;

int main(){
	printf("Введіть значення x:\n");
	while(scanf("%lf",&x)==0){
		double x1;
		while((x1=getchar())!='\n' && x1!=EOF);
		printf("Неприйнятні дані (x має бути числом):");
	}

	printf("Введіть значення y:\n");
	while(scanf("%lf",&y)==0){
		double y1;
		while((y1=getchar())!='\n' && y1!=EOF);
		printf("Неприйнятні дані (y має бути числом):");
	}

	printf("Введіть значення z:\n");
	while(scanf("%lf",&z)==0){
		double z1;
		while((z1=getchar())!='\n' && z1!=EOF);
		printf("Неприйнятні дані (z має бути числом):");
	}

	i=pow(y,-sqrt(fabs(x)));

	if(i<=0){
		printf("Функція невизначена!\n");
	}

	else{
		r=log(i)*(x-y/2)+pow(sin(atan(z)),2); printf("%lf\n",r); }
	return 0;
}

\end{lstlisting}

{\fontsize{14}{16.1}\selectfont
Вивід:}

\begin{lstlisting}

\end{lstlisting}

\section{Програмний код до другого завдання}

\begin{lstlisting}[frame=single]

#include <stdio.h>
#include <math.h>
#include <stdlib.h>

double x,y,z,i,r,num;
int func2(){
	printf("Введіть значення z:\nz = ");
	scanf("%lf",&z);

	x=(z>=-1)?fabs(z):-z/3;

	if(x==-2.5){
		printf("x = -2.5, функція невизначена (логарифм нуля)\n");
		exit(0);

	}

	r=x-log(x+2.5)+3*(exp(x)-exp(-x));
	printf("Результат обчислень: %lf\n",r);

	return 0;
}

int func1(){
	printf("Введіть значення x:\n");
	scanf("%lf",&x);

	printf("Введіть значення y:\n");
	scanf("%lf",&y);

	printf("Введіть значення z:\n");
	scanf("%lf",&z);

	i=pow(y,-sqrt(fabs(x)));

	if(i<=0){
		printf("Функція невизначена!\n");
		exit(0);
	}

	else{
	r=log(i)*(x-y/2)+pow(sin(atan(z)),2);
	printf("%lf\n",r);
	}

	return 0;
}

int main(){
	printf("Номер функції для подальшої роботи (1 або 2): ");
	scanf("%lf",&num);

	if(num==1)
		func1();
	if(num==2)
		func2();
}

\end{lstlisting}

{\fontsize{14}{16.1}\selectfont
Вивід:}

\begin{lstlisting}

\end{lstlisting}

{\fontsize{14}{16.1}\selectfont
\textbf{Висновок:}

Висновок: виконуючи цю лабораторну роботу, я застосував оператори умовного переходу на
практиці та краще зрозумів програмування алгоритмів розгалуженої структури.

}

\end{document}
