%xelatex
\documentclass{article}

%\usepackage[T2A]{fontenc}
\usepackage[english,ukrainian]{babel}
\usepackage{fontspec}
\setmainfont{Nimbus Roman}
\usepackage{graphicx}
\usepackage[a4paper,margin=0.5in]{geometry}
\usepackage{amsmath}
\begin{document}

\pagestyle{empty}
\begin{center}

	{\fontsize{14}{24}\selectfont МІНІСТЕРСТВО ОСВІТИ І НАУКИ УКРАЇНИ

	НАЦІОНАЛЬНИЙ УНІВЕРСИТЕТ «ЛЬВІВСЬКА ПОЛІТЕХНІКА»

	Інститут комп'ютерних наук та інформаційних технологій

	Кафедра ОМП

	}

	\vspace{90.4pt} %120.4+16.1
	\begin{figure}[h]
		\centering
		\includegraphics[width=6.5cm,keepaspectratio]{../lpnu.png}
	\end{figure}

	{\fontsize{18}{29}\selectfont{Звіт}

	{до розрахункової роботи №1}

	{З дисципліни}

	{``Алгебра та геометрія''}

	{Варіант 12}

	}
\end{center}

\vspace{12.1pt} %30.1pt
	{\fontsize{14}{22.4}\selectfont
\begin{flushright}
	\textit{Виконав:}

	\textit{студент групи ПП-14}

	\textit{Мілюхін Олександр}

	\textit{Прийняв:}

	\textit{Баранецький Я. О.}
\end{flushright}
\vspace{37.4pt} %100.4
\begin{center}
\textit{Львів-2022}
\vspace{37.4pt} %100.4
\end{center}
	}
{\fontsize{14}{16.1}\selectfont
\begin{enumerate}
%завд1
\item Знайти добуток матриць A та B
\begin{enumerate}
\item
\begin{equation}
	A=\begin{bmatrix}
		3&2&-1\\
		1&-2&5\\
	\end{bmatrix},
	B=\begin{bmatrix}
		1&-3&1\\
		0&9&5\\
		1&7&2\\
	\end{bmatrix}.
\end{equation}
Розв'язок:
\begin{equation}
	A \times B=\begin{bmatrix}
		3+0-1&-9+18-7&3+10-2\\
		1+0+5&-3-18+35&1-10+10\\
	\end{bmatrix}=
	\begin{bmatrix}
		2&2&11\\
		6&14&1\\
	\end{bmatrix}
\end{equation}

\item
\begin{equation}
		A=\begin{bmatrix}
			-1&2&13\\
			2&-1&1\\
			3&5&-1
		\end{bmatrix},
		B=\begin{bmatrix}
			3&2&-1&0\\
			1&2&-5&4\\
			4&-1&4&3
		\end{bmatrix}.
\end{equation}
Розв'язок:
\begin{equation}
%\begin{multline}
\begin{split}
	A \times B=
		\begin{bmatrix}
			-3+2+52&-2+4-13&1-10+52&0+8+39\\
			6-1-4&4-2+1&-2+5-4&0-4-3\\
			3+5-4&6+10+1&-3-25-4&0+20-3
		\end{bmatrix} =
		\\
		=
		\begin{bmatrix}
			51&-11&43&47\\
			1&3&-1&-7\\
			10&17&-32&17
		\end{bmatrix}
%\end{multline}
\end{split}
\end{equation}
\end{enumerate}
%фін1
Відповідь: а) $
		\begin{bmatrix}
			2&2&11\\
			6&14&1\\
		\end{bmatrix}$, б) $
		\begin{bmatrix}
			51&-11&43&47\\
			1&3&-1&-7\\
			10&17&-32&17
		\end{bmatrix}$
%завд2
\item Обчислити визначники
\begin{enumerate}
	\item
		\begin{equation}
			\Delta = \begin{vmatrix}
				6&-3&1\\
				1&5&-3\\
				2&-1&4
			\end{vmatrix}
		\end{equation}
Розв'язок:
		\begin{equation}
			\begin{vmatrix}
				6&-3&1\\
				1&5&-3\\
				2&-1&4
			\end{vmatrix} = 120-1+18-10-18+12=121.
		\end{equation}
	\item
		\begin{equation}
			\Delta = \begin{vmatrix}
				6&-3&1&4\\
				1&5&-3&3\\
				2&-1&4&5\\
				7&2&-2&7
			\end{vmatrix}
		\end{equation}
Розв'язок:
		\begin{equation}
			\begin{split}
			\begin{vmatrix}
				6&-3&1&4\\
				1&5&-3&3\\
				2&-1&4&5\\
				7&2&-2&7
			\end{vmatrix}= 7 \cdot (-1)^{4+1} \cdot
			\begin{vmatrix}
				-3&1&4\\
				5&-3&3\\
				-1&4&5\\
			\end{vmatrix} + 2 \cdot (-1)^{4+2} \cdot
			\begin{vmatrix}
				6&1&4\\
				1&-3&3\\
				2&4&5\\
			\end{vmatrix} \\ -2 \cdot (-1)^{4+3} \cdot
			\begin{vmatrix}
				6&-3&4\\
				1&5&3\\
				2&-1&5\\
			\end{vmatrix} + 7 \cdot (-1)^8 \cdot 121 = 7 \cdot (-1) \cdot ((-3 \cdot (-3) \cdot 5-3+80) - \\ - (12+25-36)) + 2 \cdot (-121) + 2 \cdot (-1) \cdot (-121) + 7 \cdot 121 = 0.
			\end{split}
		\end{equation}
\end{enumerate}
%фін2
Відповідь: а) 121, б) 0.
%завд3
	\item Розв'язати систему лінійних рівнянь:
\begin{enumerate}
	\item методом Крамера;
	\item матричним методом.
\end{enumerate}
\begin{equation}
\begin{cases}
	x_1-x_2+x_3=1\\
	2x_1+x_2-x_3=2\\
	2x_1+5x_2-3x_3=4\\
\end{cases}
\end{equation}
Розв'язок:
\begin{enumerate}
	\item
	\item
\end{enumerate}
%завд4
\item Знайти матрицю X з матричного рівняння $$A \cdot X \cdot B=D, $$якщо$$ D=3C-2B+A$$
%завд5
\item Дослідити на сумісність системи лінійних рівнянь. У випадку сумісності системи знайти її розв'язок.
%завд6
\item Розв'язати однорідні системи рівнянь
%завд7
\item Знайти скалярний добуток векторів $\vec{m}=2\vec{a}+\vec{b}$ і $\vec{n}=\vec{a}-3\vec{b}$, косинус кута між векторами $\vec{m}$ і $\vec{n}$ і пр$_{\vec{n}}\vec{m}.$
%завд8
\item Вершини піраміди знаходяться в точках $A_1, A_2, A_3, A_4.$ Знайти площу грані $A_1, A_2, A_3,$ довжину висоти піраміди, проведеної з вершини $A_4$, і об'єм піраміди.
%завд9
\item Перевірити, чи компланарні вектори $\vec{a}, \vec{b}, \vec{c}.$
%завд10
\item Сили $\vec{F_1},\vec{F_2}$ і $\vec{F_3}$ прикладені до точки $M_1$. Обчислити роботу, яку виконує рівнодійна цих сил під час переміщення матеріальної точки з положення $M_1$ у положення $M_2$ по відрізку прямої.
%завд11
\item Сила $\vec{F}$ прикладена до точки A. Знайти момент $\vec{M}$ сили $\vec{F}$ відносно точки B.
%завд21
\item[21.] Довести, що вектори $\vec{e_1},\vec{e_2},\vec{e_3}$ утворюють базис та знайти координати вектора $\vec{x}$ у цьому базисі.
%завд23
\item[23.] Знайти власні значення і власні вектори лінійних перетворень, які задані матрицею A, і записати матрицю перетворення A' в базисі з власних векторів.
\end{enumerate}

%\textbf{Висновок:}
%<>
}

\end{document}
