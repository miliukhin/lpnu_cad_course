%xelatex
\documentclass[14pt,draft]{extreport}

%\usepackage[T2A]{fontenc}
\usepackage[english,ukrainian]{babel}
%\usepackage{fontspec}
\usepackage{amsmath}
\usepackage{mathspec}
\setallmainfonts{Nimbus Roman}
\usepackage{graphicx}
\usepackage[a4paper,margin=0.5in]{geometry}
\usepackage{pgfplots}
\pgfplotsset{width=10cm,compat=1.8}
\usepackage{indentfirst}
\usepackage{wrapfig}
\usepackage{boxedminipage}
\usepackage{spalign}
\usepackage{subfiles}

\begin{document}

\bigskip \textbf{Задача 14.} Знайти півосі, фокуси, ексцентриситет,
координати центра та рівняння директрис $5x^2+3y^2-10x-6y-7=0.$\bigskip

%\begin{wrapfigure}{l}{5.3cm}
%	\centering
%\subfile{../g/14.tex}
%\caption{}
%\end{wrapfigure}
\begin{equation}
\begin{aligned}
5x^2+3y^2-10x-6y-7=0;\\
	5(x^2-2x+1)-5+3(y^2-2y+1)-3-7=0;\\
	5(x-1)^2+3(y-1)^2=15;\\
	\frac{(x-1)^2}{\frac{15}{5}}+\frac{(y-1)^2}{\frac{15}{3}}=1
	\rightarrow
	\frac{(x-1)^2}{3}+\frac{(y-1)^2}{5}=1\\
\end{aligned}
\end{equation}

Таким чином отримали канонічне рівняння еліпса.
Його мала піввісь $a=\sqrt{3},$
а велика~--- $b=\sqrt{5}$. Координати центра~---
$(1;1).$

Знайдемо половину відстані між фокусами:

\begin{equation}
	b^2-c^2=a^2\to c=\sqrt{b^2-a^2}=\sqrt{5-3}=\sqrt{2}
\end{equation}

Тоді самі фокуси мають координати:
$$F_1=(1-\sqrt{2};1),~F_2=(1+\sqrt{2};1).$$

Ексцентриситет даного еліпса дорівнює:

\begin{equation}
	\varepsilon=\frac{c}{a}=\frac{\sqrt{2}}{\sqrt{3}}
\end{equation}

%\begin{equation}
%	\frac{r}{d}=\varepsilon=\frac{\sqrt{2}}{\sqrt{3}}
%\end{equation}

Звідси рівняння директрис:
\begin{equation}
	x=\pm\frac{a}{\varepsilon}=\pm\frac{\sqrt{3}}
	{\frac{\sqrt{2}}{\sqrt{3}}}=\pm\frac{3}{\sqrt{2}}.
\end{equation}

\null\hfill
\begin{boxedminipage}{0.4\textwidth}
	Відповідь: $b=\sqrt{5}, a=\sqrt{3};\\
	$Центр еліпса~--- $O(1;1);$\\
	$F_1=(1-\sqrt{2};1),~F_2=(1+\sqrt{2};1).$\\
	$\varepsilon=\frac{\sqrt{2}}{\sqrt{3}}$\\
	Рівняння директрис: $x=\pm\frac{3}{\sqrt{2}}.$
\end{boxedminipage}

\end{document}
