%xelatex
\documentclass[14pt,draft]{extreport}

%\usepackage[T2A]{fontenc}
\usepackage[english,ukrainian]{babel}
%\usepackage{fontspec}
\usepackage{amsmath}
\usepackage{mathspec}
\setallmainfonts{Nimbus Roman}
\usepackage{graphicx}
\usepackage[a4paper,margin=0.5in]{geometry}
\usepackage{pgfplots}
\pgfplotsset{width=10cm,compat=1.8}
\usepackage{indentfirst}
\usepackage{wrapfig}
\usepackage{boxedminipage}
\usepackage{subfiles}

\begin{document}

\bigskip \textbf{Задача 19.} Знайти точку $P'$, яка симетрична
до точки $P(5;2;1)$ відносно площини $-x+2y-3z+10=0$. \bigskip

%\begin{wrapfigure}{l}{5.3cm}
%	\centering
%\subfile{../g/19.tex}
%\caption{}
%\end{wrapfigure}

Запишемо параметричні рівняння прямої, що проходить через $P$
перпендикулярно до заданої площини та підставимо отримані
значення $x, y, z$ у рівняння площини:
\begin{equation}
	x=5-t,~y=2+2t,~z=1-3t,
\end{equation}
\begin{equation}
	-1(5-t)+2(2+2t)-3(1-3t)+10=0.
\end{equation}

Звідси $t=-\frac{3}{7},$ а координати точки $P_0$
(проєкції $P$ на площину): $~x_{P_0}=5\frac{3}{7},
~y_{P_0}=1\frac{1}{7}~,z=2\frac{2}{7}.$ Тепер можна
знайти координати $P',$ застосувавши формулу
середини відрізка:
\begin{equation}
	5\frac{3}{7}=\frac{5+x_{P'}}{2}\rightarrow
	x_{P'}=\frac{41}{7},
	1\frac{1}{7}=\frac{2+y_{P'}}{2}\rightarrow
	y_{P'}=\frac{2}{7},~
	2\frac{2}{7}=\frac{1+z_{P'}}{2}\rightarrow
	z_{P'}=\frac{25}{7}.
\end{equation}


\null\hfill
\begin{boxedminipage}{0.73\textwidth}
	Відповідь: $P'(\frac{41}{7}, \frac{2}{7}, \frac{25}{7}).$
\end{boxedminipage}

\end{document}
