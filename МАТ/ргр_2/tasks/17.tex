%xelatex
\documentclass[14pt,draft]{extreport}

%\usepackage[T2A]{fontenc}
\usepackage[english,ukrainian]{babel}
%\usepackage{fontspec}
\usepackage{amsmath}
\usepackage{mathspec}
\setallmainfonts{Nimbus Roman}
\usepackage{graphicx}
\usepackage[a4paper,margin=0.5in]{geometry}
\usepackage{pgfplots}
\pgfplotsset{width=10cm,compat=1.8}
\usepackage{indentfirst}
\usepackage{wrapfig}
\usepackage{boxedminipage}
\usepackage{subfiles}

\begin{document}

\bigskip \textbf{Задача 17.} Звести загальне рівняння прямої
$\begin{cases}
	4x+y+z+2=0\\
	2x-y-3z-8=0\\
\end{cases}$
до канонічного вигляду і записати її параметричне
рівняння.\bigskip

%\begin{wrapfigure}{l}{5.3cm}
%	\centering
%\subfile{../g/17.tex}
%\caption{}
%\end{wrapfigure}

Знайдемо напрямний вектор $\vec s$ цієї прямої
як векторний добуток нормальних векторів площин,
що задають цю пряму:
\begin{equation}
	\vec s = \begin{vmatrix}
		\vec i & \vec j & \vec k\\
		4 & 1 & 1\\
		2 & -1 & -3\\
	\end{vmatrix}
	=(-2;14;-6).
\end{equation}

Пряма проходить через $M_0(x_0=0;y_0;z_0),$
знайдемо інші координати цієї точки:

\begin{equation}
\begin{cases}
	y+z+2=0\\
	-y-3z-8=0\\
\end{cases}
\to
\begin{cases}
	y-3+2=0\to y=1\\
	-2z-6=0\to z=-3\\
\end{cases}
\end{equation}

Отже, канонічне рівняння цієї прямої має вигляд:
\begin{equation}
	\frac{x}{-2}=\frac{y-1}{14}=\frac{z+3}{-6},
\end{equation}
а параметричне~---
\begin{equation}
	\begin{cases}
		\frac{x-x_0}{m}=t\to x=x_0+mt=0-2t;\\
		\frac{y-y_0}{m}=t\to y=y_0+mt=1+14t;\\
		\frac{z-z_0}{m}=t\to z=z_0+mt=3-6t.\\
	\end{cases}
\end{equation}

\null\hfill
\begin{boxedminipage}{0.34\textwidth}
	Відповідь:
	$
	\begin{cases}
		x=0-2t;\\
		y=1+14t;\\
		z=3-6t.\\
	\end{cases}
	$
\end{boxedminipage}

\end{document}
