%xelatex
\documentclass[14pt,draft]{extreport}

%\usepackage[T2A]{fontenc}
\usepackage[english,ukrainian]{babel}
%\usepackage{fontspec}
\usepackage{mathtools}
\usepackage{amsmath}
\usepackage{mathspec}
\setallmainfonts{Nimbus Roman}
\usepackage{graphicx}
\usepackage[a4paper,margin=0.5in]{geometry}
\usepackage{pgfplots}
\pgfplotsset{width=10cm,compat=1.8}
\usepackage{indentfirst}
\usepackage{wrapfig}
\usepackage{boxedminipage}
\usepackage{subfiles}

\begin{document}

\bigskip \textbf{Задача 15.} Скласти рівняння площини, що проходить
через точки $M_1(1;1;-2)$ і $M_2(3;-2;1)$ паралельно до вектора
$\vec{a}(0;3;5)$.\bigskip

%\begin{wrapfigure}{l}{5.3cm}
%	\centering
%\subfile{../g/15.tex}
%\caption{}
%\end{wrapfigure}

Складемо рівняння прямої, що проходить через $M_1$ та $M_2$:
\begin{equation}
	\frac{x-1}{2}=
	\frac{y-1}{-3}=
	\frac{z+2}{3}.
\end{equation}
Її напрямний вектор компланарний до $\vec a$, отже,
їх мішаний добуток дорівнює нулю:
\begin{equation}
	\begin{aligned}
	0=\begin{vmatrix}
		x-1 & y-1 & z+2\\
		2 & -3 & 3\\
		0 & 3 & 5\\
	\end{vmatrix}=
	(x-1)\begin{vmatrix}
		-3 & 3 \\ 0 & 3
	\end{vmatrix}-
	(y-1)\begin{vmatrix}
		2 & 3 \\ 0 & 5
	\end{vmatrix}-
	(z+2)\begin{vmatrix}
		2 & -3 \\ 0 & 3
	\end{vmatrix}=\\
	=-26x-10y+6z+48=0.
	\end{aligned}
\end{equation}
\null\hfill
\begin{boxedminipage}{0.5\textwidth}
{Відповідь:} $-26x-10y+6z+48=0.$
\end{boxedminipage}\\

\end{document}
