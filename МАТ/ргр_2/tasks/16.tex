%elatex
\documentclass[14pt,draft]{extreport}

%\usepackage[T2A]{fontenc}
\usepackage[english,ukrainian]{babel}
%\usepackage{fontspec}
\usepackage{amsmath}
%\usepackage{mathspec}
%\setallmainfonts{Nimbus Roman}
\usepackage{graphicx}
\usepackage[a4paper,margin=0.5in]{geometry}
\usepackage{pgfplots}
\pgfplotsset{width=10cm,compat=1.8}
\usepackage{indentfirst}
\usepackage{wrapfig}
\usepackage{boxedminipage}
\usepackage{subfiles}

\begin{document}

\bigskip \textbf{Задача 16.} Знайти:
\begin{enumerate}
	\item рівняння площини $\pi$, що проходить через три точки
		$M_1(1;3;6),~M_2(2;2;1),\\ M_3(-1;0;1)$;
	\item відстань від точки $M_0(5;-4;5)$ до площини $\pi$;
	\item рівняння прямої $l,$ що проходить через точки $M_0$ і $M_1$.
\end{enumerate}
		\bigskip

%\begin{wrapfigure}{l}{5.3cm}
%	\centering
%\subfile{../g/16.tex}
%\caption{}
%\end{wrapfigure}

Візьмемо біжучу точку $M(x;y;z)$ на площині $\pi$ і побудуємо вектори
$\vec{M_1M},\vec{M_1M_2}\\\text{та}~\vec{M_1M_3},$
за умови компланарності яких ця площина буде визначена.
Отже, їх мішаний добуток дорівнює нулю:

\begin{equation}
\begin{aligned}
	0=
	\begin{vmatrix}
		x-1 & y-3 & z-6\\
		2-1 & 2-3 & 1-6\\
		-1-1& 0-3 & 1-6\\
	\end{vmatrix}
	=
	\begin{vmatrix}
		x-1 & y-3 & z-6\\
		1 & -1 & -5\\
		-2& -3 & -5\\
	\end{vmatrix}
	=\\
	= (x-1)(5-15)-(y-3)(-5-10)+(z-6)(-3-2)=\\
	-10x+10+15y-45-5z+30.
\end{aligned}
\end{equation}
Рівняння площини $\pi$:
\begin{equation}
	-10x+15y-5z-5=0.
\end{equation}

2. Відстань від точки $M_0(5;-4;5)$ до площини $\pi$;

\begin{equation}
	d(M_0,\pi)=\frac{|Ax_0+By_0+Cz_0+D|}{\sqrt{A^2+B^2+C^2}}=
	\frac{|-10\cdot5-15\cdot4-5\cdot5-5|}{\sqrt{100+225+25}}
\end{equation}

\begin{equation}
	d(M_0,\pi)=\frac{140}{\sqrt{350}=5\sqrt{14}}
\end{equation}

3. Рівняння прямої, що проходить через точки $M_0~\text{і}~M_1$:

\begin{equation}
	\frac{x-x_0}{x_1-x_0}=...,~\text{або}~
	\frac{x-5}{1-5}=\frac{y+4}{3+4}=\frac{z-5}{6-5}=
	\frac{x-5}{-4}=\frac{y+4}{7}=\frac{z-5}{1}.
\end{equation}

\null\hfill
\begin{boxedminipage}{0.33\textwidth}
	Відповідь:
	\begin{enumerate}
		\item $-10x+15y-5z=0$.
		\item $\frac{140}{5\sqrt{14}}$.
		\item $\frac{x-5}{-4}=\frac{y+4}{7}=\frac{z-5}{1}$.
	\end{enumerate}
\end{boxedminipage}

\end{document}
