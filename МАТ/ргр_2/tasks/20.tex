%xelatex
\documentclass[14pt,draft]{extreport}

%\usepackage[T2A]{fontenc}
\usepackage[english,ukrainian]{babel}
%\usepackage{fontspec}
\usepackage{amsmath}
\usepackage{mathspec}
\setallmainfonts{Nimbus Roman}
\usepackage{graphicx}
\usepackage[a4paper,margin=0.5in]{geometry}
\usepackage{pgfplots}
\pgfplotsset{width=10cm,compat=1.8}
\usepackage{indentfirst}
\usepackage{wrapfig}
\usepackage{boxedminipage}
\usepackage{subfiles}

\begin{document}

\bigskip \textbf{Задача 20.} Знайти проекцію точки $M(0;2;-3)$
на пряму
$\frac{x}{5}=\frac{y-1}{7}=\frac{z-2}{1}.$\bigskip

%\begin{wrapfigure}{l}{5.3cm}
%	\centering
%\subfile{../g/20.tex}
%\caption{}
%\end{wrapfigure}

Проведемо площину $\alpha$, що проходить через точку $M$
перпендикулярно до прямої
$\frac{x}{5}=\frac{y-1}{7}=\frac{z-2}{1}.$
Її нормальний вектор $\vec n(5;7;1)$
~--- напрямний для прямої:
\begin{equation}
	5(x-0)+7(y-2)+(z+3)=0, \text{або}~
	5x+7y+z-11=0.
\end{equation}

Точка, яка є шуканою проєкцією, повинна належати
і цій площині, і прямій, тому складемо систему
рівнянь та розв'яжемо її, перевівши рівняння прямої
в параметричне:
\begin{equation}
\begin{cases}
	\frac{x}{5}=\frac{y-1}{7}=\frac{z-2}{1};\\
	5x+7y+z-11=0.
\end{cases}\to
\begin{cases}
	x=5t;\\
	y=7t+1;\\
	z=t+2.\\
	25t+49t+7+t+2-11=0;\\
\end{cases}\to
\begin{cases}
	x=\frac{10}{75};\\
	y=1\frac{14}{75};\\
	z=2\frac{2}{75}.\\
\end{cases}
\end{equation}

Отже, $(\frac{10}{75};
	1\frac{14}{75};
	2\frac{2}{75})$~---
координати проєкції точки $M$ на дану пряму.

\null\hfill
\begin{boxedminipage}{0.32\textwidth}
	Відповідь:~$(\frac{10}{75};
	1\frac{14}{75};
	2\frac{2}{75})$.
\end{boxedminipage}

\end{document}
