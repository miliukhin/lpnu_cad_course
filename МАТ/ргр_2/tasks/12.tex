%xelatex
\documentclass[14pt,draft]{extreport}

%\usepackage[T2A]{fontenc}
\usepackage[english,ukrainian]{babel}
%\usepackage{fontspec}
\usepackage{amsmath}
\usepackage{mathspec}
\setallmainfonts{Nimbus Roman}
\usepackage{graphicx}
\usepackage[a4paper,margin=0.5in]{geometry}
\usepackage{pgfplots}
\pgfplotsset{width=10cm,compat=1.8}
\usepackage{indentfirst}
\usepackage{wrapfig}
\usepackage{subfiles}
\usepackage{boxedminipage}

\begin{document}

\textbf{Задача 12.} Написати рівняння медіани і висоти, які
проведені в трикутнику ABC з вершини B. \medskip

\begin{wrapfigure}{l}{5.3cm}
	\centering
\subfile{../g/12.tex}
\caption{}
\end{wrapfigure}

$A(3;5),~B(3;1),~C(-5;7).$\medskip

Рівняння медіани можна записати в канонічному вигляді,
узявши як напрямний вектор від вершини B до середини
протилежної їй сторони, $AC$, $M_0$, координати якої:

\begin{equation}
	x_{M_0}=\frac{3-5}{2}=-1,~
	y_{M_0}=\frac{5+7}{2}=6,
\end{equation}

а координати напрямного вектора $\vec{S}(m,n)=\vec{BM_0}=(-1-3;6-1)=(-4;5)$.
Отже, рівняння виглядатиме так:
\begin{equation}
	\frac{x-x_0}{m}=\frac{y-y_0}{n},~\text{або}~
	\frac{x+1}{-4}=\frac{y-6}{5}.
\end{equation}

Рівняння висоти отримаємо таким самим способом, узявши
за її напрямний вектор $\vec{S'}$~--- нормальний для прямої
$\vec{AC}$, рівняння якої:

\begin{equation}
	\frac{x-3}{-8}=\frac{y-5}{2} \rightarrow
	x-3=-4y+20 \rightarrow x+4y-23=0.
\end{equation}

Отже, (1,4) - координати шуканого вектора $\vec{S'}$. Рівняння висоти:

\begin{equation}
	\frac{x-3}{1}=\frac{y-1}{4}
\end{equation}

\medskip

\null\hfill
\begin{boxedminipage}{0.73\textwidth}
Відповідь: р-ня медіани - $\frac{x+1}{-4}=\frac{y-6}{5},$
р-ня висоти: $\frac{x-3}{1}=\frac{y-1}{4}.$
\end{boxedminipage}

\end{document}
