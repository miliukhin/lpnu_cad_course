%xelatex
\documentclass[14pt,draft]{extreport}

%\usepackage[T2A]{fontenc}
\usepackage[english,ukrainian]{babel}
%\usepackage{fontspec}
\usepackage{amsmath}
\usepackage{mathspec}
\setallmainfonts{Nimbus Roman}
\usepackage{graphicx}
\usepackage[a4paper,margin=0.5in]{geometry}
\usepackage{pgfplots}
\pgfplotsset{width=10cm,compat=1.8}
\usepackage{indentfirst}
\usepackage{wrapfig}
\usepackage{boxedminipage}
\usepackage{subfiles}

\begin{document}

\bigskip \textbf{Задача 18.} Знайти відстань від точки $P(1;-3;1)$
до прямої $\frac{x}{5}=\frac{y+1}{2}=\frac{z-1}{1}.$\bigskip

\begin{wrapfigure}{l}{5.3cm}
	\centering
\subfile{../g/18.tex}
\caption{}
\end{wrapfigure}

Відстань від точки P до заданої прямої~--- довжина висоти паралелограма,
побудованого на векторах $\vec{M_0P}=(1-0;-3+1;1-1)$ та $\vec s=(5;2;1)$.

Знаючи, що площа паралелограма дорівнює модулю векторного добутку його
сусідніх сторін, а також добутку його сторони на висоту, проведену до
неї, отримаємо співвідношення:

\begin{equation}
	{|\vec s|}d=|[\vec{M_0P},\vec s]|;~
	d=\frac{|[\vec{M_0P},\vec s]|}{|\vec s|}
	\label{S}
\end{equation}

Виконаємо необхідні обчислення:

\begin{equation}
	[\vec{M_0P},\vec s]=
	\begin{vmatrix}
		\vec i & \vec j & \vec k \\
		1 & -2 & 0 \\
		5 & 2 & 1 \\
	\end{vmatrix}=
	-2\vec i -1\vec j -8\vec k  = (-2;-1;8).
\end{equation}

\begin{equation}
	|[\vec{M_0P},\vec s]|=\sqrt{4+1+64}=\sqrt{69},~
	|\vec s| = \sqrt{25+4+1}=\sqrt{30}
\end{equation}

підставивши ці значення у (\ref{S}), знайдемо відстань:

\begin{equation}
	d=\frac{\sqrt{69}}{\sqrt{30}}=\frac{\sqrt{69}\sqrt{30}}{30}=
	\frac{\sqrt{46}}{10}.
\end{equation}

\null\hfill
\begin{boxedminipage}{0.27\textwidth}
	Відповідь: $d=\frac{\sqrt{46}}{10}.$
\end{boxedminipage}

\end{document}
