%-shell-escape, якщо використовуєте minted
\documentclass[a4paper, 12pt, oneside]{extarticle}
\input{$HOME/Templates/lpnu_doc_templates/settings/preamble.tex}
% якщо домахуються за Times New Roman, то
% використовуєте xelatex і цей файл:
\input{$HOME/Templates/lpnu_doc_templates/settings/font_styles.tex}
\input{$HOME/Templates/lpnu_doc_templates/settings/minted_settings.tex}

\newcommand\Variant{4}
\newcommand\Date{22.04.\the\year}
\newcommand\Discipline{Об'єктно-орієнтоване програмування}
\newcommand\Instructor{Патерега Ю. І.}

\newcommand\Type{\Lab}
\newcommand\Number{4}
\newcommand\Topic{Обробка виняткових ситуацій}

\begin{document}
\Margins

\Margins
%\begin{wrapfigure}[3]{l}{.27\textwidth}
%\includegraphics[width=.28\textwidth]{$UNI/.templates/lpnu_logo.png}
%\end{wrapfigure}

%\noindent\textbf{Прізвище:} \Lname \\
%\noindent\textbf{Ім'я:} \Fname \\
%\noindent\textbf{Група:} \Group \\
%\noindent\textbf{Варіант:} \Variant \\
%\noindent\textbf{Дата захисту:} \Date \\
%\\
%\noindent\textbf{Кафедра:} \Department \\
%\noindent\textbf{Дисципліна:} \Discipline \\
%\noindent\textbf{Перевірив:} \Instructor \\

%%\medskip\bigskip

%\begin{center}
%	\textbf{ЗВІТ}		\\
%	до \Type~\No\Number	\\
%	на тему ``\Topic''	\\
%\end{center}

% \begin{table}
%   \begin{tabularx}{\textwidth}{|c|X|X|}
%     \hline
%     % Image & Content & Additional Info \\
%     % \hline
% 	  \multirow{3}{*}{\includegraphics[width=4cm]{$UNI/.templates/lpnu_logo.png}}
% 	  & \textbf{ЗВО:}
% 	  Національний університет ``Львівська Політехніка''.
% 	  & \textbf{Тема:}
% 	  \Topic
% 	  \\
% 	  & \textbf{Навчальний рік:}
% 	  2023/2024
% 	  & \textbf{Інститут}
% 	  комп'ютерних наук та інформаційних технологій
% 	  \\
% 	  & \textbf{Семестр:}
% 	  осінній
% 	  & \textbf{Група:}
% 	  \Group
% 	  \\
% 	  & \textbf{Навчальна дисципліна:}
% 	  \Discipline
% 	  & \textbf{Студент:}
% 	  Мілюхін Олександр
% 	  \\
% 	  & \textbf{Кафедра}
% 	  систем автоматизованого проектування
% 	  &
% 	  \\
% 	  & \textbf{Викладач:}
% 	  Чумакевич В. В.
% 	  &
% 	  \\
%     \hline
%   \end{tabularx}
% \end{table}

\setlength{\textfloatsep}{-16pt}
% \setlength{\intextsep}{0pt}

\begin{table}
	\begin{tabular}{|l|l|p{6cm}|}
    \hline
    % Image & Content & Additional Info \\
    % \hline
	  \makecell[l]{
	  \includegraphics[width=3.37cm]{$UNI/.templates/lpnu_logo.png}
  }
	  & \makecell[l]{
	  \textbf{ЗВО:}
	  Національний університет \\ ``Львівська Політехніка''.
	  \\
	  \textbf{Навчальний рік:}
	  2023/2024
	  \\
	  \textbf{Семестр:}
	  осінній
	  \\
	  \textbf{Навчальна дисципліна:} \\
	  \Discipline
	  \\
	  \textbf{Кафедра}
	  систем автоматизованого \\ проектування
	  \\
	  \textbf{Викладач:}
	  Чумакевич В. В.
}
	  & \makecell [l] {
	  \textbf{Тема:}
	  \Topic
	  \\
          \textbf{Інститут}
	  комп'ютерних наук та \\ інформаційних технологій
	  \\
	  \textbf{Група:}
	  \Group
	  \\
	  \textbf{Студент:}
	  Мілюхін Олександр
  }
  \\
    \hline
  \end{tabular}
\end{table}
\section{Мета роботи}

% \begin{table}
%   \begin{tabularx}{\textwidth}{|p{6cm}|c|c|}
% 	  \hline
%     \multirow{3}{*}{\includegraphics[width=6cm]{$UNI/.templates/lpnu_logo.png}}
% 	  & ЗВО: Національний університет ``Львівська Політехніка''
% 	  & Additional Info 1 \\
%     & Content 2 & Additional Info 2 \\
%     & Content 3 & Additional Info 3 \\
% 	  \hline
%   \end{tabularx}
% \end{table}

% \begin{table}
%   \begin{tabular}{|c|c|c|}
%     \hline
%     \multirow{3}{*}{\includegraphics[width=3cm]{$UNI/.templates/lpnu_logo.png}} & \makecell{Content 1 \\ Content 2 \\ Content 3} & \makecell{Additional Info 1 \\ Additional Info 2 \\ Additional Info 3} \\
%     \hline
%   \end{tabular}
% \end{table}


Набути уміння та навички розробки та опису програм з використанням
обробки виняткових (виключних) ситуацій.

\section*{Індивідуальне завдання}

\subsection*{Завдання 1}

4. Визначити обробник виняткової ситуації, яка виникає при перетворенні
об’єкта класу у значення цілочисельного типу.

\subsection*{Завдання 2}

4. Скласти програму знаходження середнього арифметичного додатніх чисел  з
використанням обробки виключних ситуацій для випадків коли додатні числа
відсутні.

\section*{Етапи розв'язку}

\subsection*{Завдання 1}

Для перехоплення такої виняткової ситуації використав
функції бібліотеки boost

\subsection*{Завдання 2}

Вирішив зберігати числа у векторі.
Спочатку реалізував цикл із перевіркою
елементів вектора на наявність додатних
чисел, а потім вирішив одразу їх додавати.
Пізніше зробив так, щоб елементи вектора можна
було вводити як аргументи через командний рядок.

\section*{Коди програм}

\subsection*{Код програми (завдання 1)}
\inputminted{c++}{task_1/task1.cpp}
\subsection*{Результат виконання програми}
\verbatiminput{task_1/out}

\subsection*{Код програми (завдання 2)}
\inputminted{c++}{task_2/avg.cpp}
\subsection*{тест}
\inputminted{sh}{task_2/test}
\subsection*{Результат виконання програми}
\inputminted{sh}{task_2/out}

\section*{Висновок}

Розібрався з основами обробки виняткових ситуацій засобами
мови c++.

\section*{Відповіді на контрольні запитання}
\begin{itemize}
	\question Що таке виняткові ситуації?
	\answer події, що призводять до ненормальної роботи програми (помилок)

	\question Синтаксис виняткових ситуацій.
	\answer У мові C++ передбачено три ключові слова для обробки виняткових ситуацій:
		\begin{enumerate}
		\item try (контролювати)
		\item catch (ловити)
		\item throw (генерувати, породжувати, кидати, посилати, формувати)
		\end{enumerate}

	\question Як контролюються виняткові ситуації
	\answer Якщо виняткова ситуація (тобто помилка)
виникає всередині блоку try, вона «збуджується» (ключове
слово throw), перехоплюється (ключове слово catch) та обробляється.


	\question Як відбувається перехват виняткових ситуацій.
	\answer Код, який може викликати виключну ситуацію, вміщується в блок try, а блок catch містить код, який обробляє виняток. Також можна створити виключення за допомогою оператора throw.

\end{itemize}

\end{document}
