%shell-escape
\documentclass[../rgr1.tex]{subfiles}
\usetikzlibrary{patterns}

\begin{document}

\Problem{
	Дослідити методами диференціального числення функцію $
	y = \frac{1}{x^2} - \frac{1}{(x-1)^2}
	$ та побудувати її графік
}

\Solution

\paragraph{ОВ}
\subparagraph{$\frac{1}{x^2}:$} $x \in (-\infty;0)\cup(0;+\infty)$
\subparagraph{$\frac{1}{(x-1)^2}:$} $x \in (-\infty;1)\cup(1;+\infty)$
\subparagraph{$D(f):$} $x \in (-\infty;0)\cup(0;1)\cup(1;+\infty)$
\paragraph{ОЗ $E(f)$} $f \in ( -\infty; +\infty )$
\paragraph{Парність}
$
	y(-x) =
	\frac{1}{x^2} - \frac{1}{(-x-1)^2}
$, функція непарна.
	%<++>
\paragraph{Періодичність $[f(x+T) = f(x)]?$}

\begin{align}
	\frac{1}{(x+T)^2} - \frac{1}{(x+T-1)^2} =
	\frac{1}{x^2} - \frac{1}{(x-1)^2}
	%<++>
\end{align}

$T$ залежить від $x$, функція неперіодична

\paragraph{Неперервність} розглянемо границі функції в сумнівних точках

\begin{equation}
	\lim_{x\to+0}\frac{1}{x^2} - \frac{1}{(x-1)^2}
	= \lim_{x\to+0}\frac{-2x+1}{x^2(x-1)^2}
	= \lim_{x\to+0}\frac{-0-0+1}{0^2(0+0-1)^2}
	% = <++>
	= +\infty
\end{equation}
так само $\lim_{x\to-0}\frac{1}{x^2} - \frac{1}{(x-1)^2} =+\infty$,
в точці розрив 2-го роду.
\begin{equation}
	\lim_{x\to 1+0}\frac{-2x+1}{x^2(x-1)^2}
	% = \lim_{x\to 1-0}\frac{-2x+1}{x^2(x-1)^2}
	= \frac{-1}{1(1+0-1)^2}
	= -\infty
\end{equation}
\begin{equation}
	\lim_{x\to 1-0}\frac{-2x+1}{x^2(x-1)^2}
	= \frac{-1}{1(1-0-1)^2}
	= -\infty
\end{equation}
Тут теж розрив 2-го роду.

\paragraph{Точки екстремуму}
Знайдімо похідну функції.
\begin{equation}
	y = \frac{1}{x^2} - \frac{1}{(x-1)^2} \to
	f'(x) = \frac{-2}{x^3} + \frac{2}{(x-1)^3}
\end{equation}

Точки екстремуму знаходяться в місцях, де похідна функції змінює знак.
$x \neq \{1; 0\}$
\begin{equation}
	\frac{2}{x^3} = \frac{2}{(x-1)^3}
\end{equation}
	% бачимо, що
	у критичних $x=\{1;0\}$ точках функція розривна
% \begin{align}
% 	x = 0 \to f'(x) = 1 > 0 \\
% 	x = \pm \frac{\sqrt{3.5} }{2} \to f'(x)
% 	= 2 - \frac{1}{\sqrt{1-\frac{3\frac{1}{2}}{4}}}
% 	= 2 - \frac{1}{\sqrt{1-{\frac{7}{8}}}}
% 	= 2 - \frac{1}{\sqrt {\frac{1}{8}} } \\
% 	\sqrt{\frac{1}{8}} = 0.35,~
% 	2 - \frac{100}{35} \approx -0.8 < 0.
% \end{align}

\begin{align}
	-\frac{2}{\frac{1}{2^3}} + \frac{2}{(\frac{1}{2}-1)^3} = -32 < 0 \\
	-\frac{2}{2^3} + \frac{2}{(2-1)^3} = -\frac{1}{4} + \frac{2}{1} > 0 \\
	-\frac{2}{(-2)^3} + \frac{2}{(-2-1)^3} = \frac{1}{4} - \frac{2}{27} > 0
\end{align}

% Перенісши значення на числову пряму, матимемо:
\begin{figure}[h]
	\centering
	\begin{tikzpicture}
	[
		letter/.style={
			circle, minimum size=3pt, inner sep=0,
			outer sep=0, fill=black, label=below:#1
		},
		crossed/.style={
			cross out, minimum size=3pt, inner sep=0,
			outer sep=0, draw=black, label=below:#1
		},
		number/.style={fill=white, draw=red, pos=.5}
	]
		\draw (-1,0) --
			node(A)[crossed=$-\infty$,pos=-2]{}
			node(Al) [crossed=$0$, pos=-.7]{}
			node(Bl) [crossed=$1$, pos=.7]{}
			node(B)[crossed=$+\infty$,pos=2]{}
			(1,0)
		;
		\draw (-5,0) -- (-1,0);
		\draw (3,0) -- (1,0);
		\draw[]
			(Al) to[bend left=50] node[number]{-} (Bl)
			(A) to[bend right=40] node[number]{+} (Al)
			(B) to[bend left=40] node[number]{+} (Bl)
		;
	\end{tikzpicture}
	\caption{відображення похідної на числовій осі}
\end{figure}

% $x = \frac{\sqrt 3}{2}$ --- максимум,
% $x = -\frac{\sqrt 3}{2}$ --- мінімум.

\paragraph{Проміжки монотонності}

Поведінку функції на проміжку можна визначити за знаком похідної:
\subparagraph{Зростання:} $x\in \left( -\infty; 0 \right] \bigcup \left[ 1; +\infty \right)$
\subparagraph{Спадання:} $x\in \left[ 0; 1 \right]$

\paragraph{Опуклість ($f''$)}

\begin{dmath}
	f''(x)
	= \frac{6}{x^4} - \frac{6}{(x-1)^4}
	= \frac{6(x-1)^4 - 6x^4}{x^4(x-1)^4} \\
	6(x-1)^4 - 6x^4 = 0 \implies x = \frac{1}{2} \\
	(2-1)^4 - 2^4 < 0 \\
	(-2-1)^4 - (-2)^4 > 0
\end{dmath}

\begin{figure}[h]
	\centering
	\begin{tikzpicture}
	[
		letter/.style={
			circle, minimum size=3pt, inner sep=0,
			outer sep=0, fill=black, label=below:#1
		},
		crossed/.style={
			cross out, minimum size=3pt, inner sep=0,
			outer sep=0, draw=black, label=below:#1
		},
		number/.style={fill=white, draw=red, pos=.5}
	]
		\draw (-1,0) --
			node(A)[crossed=$-\infty$,pos=-2]{}
			node(Al) [letter=$\frac{1}{2}$, pos=-.5]{}
			node(B)[crossed=$+\infty$,pos=2]{}
			(1,0)
		;
		\draw (-5,0) -- (-1,0);
		\draw (3,0) -- (1,0);
		\draw[]
			(Al) to[bend right=30] node[number]{+} (A)
			(B) to[bend left=20] node[number]{-} (Al)
		;
	\end{tikzpicture}
	\caption{відображення другої похідної на числовій осі}
\end{figure}

\paragraph{$\bigcup (f'' > 0)$} $x \in (-\infty; \frac{1}{2})$
\paragraph{$\bigcap(f'' < 0)$} $x \in (\frac{1}{2}; +\infty)$

\paragraph{Асимптоти}

\subparagraph{вертикальні:}
$x = 1, x = 0$

\subparagraph{похилі:}
$y = kx + b, k = \lim_{x\to\infty}\frac{f(x)}{x},$
$ b = \lim_{x\to\infty} ( f(x) - kx )$

\begin{equation}
	k = \lim_{x\to\infty} \frac{ \frac{1}{x^2} - \frac{1}{(x-1)^2} }{x} =
	\left\{
		\frac{0-0}{\infty}
	\right\} = 0 \implies \text{вісь Ox --- асимптота}
\end{equation}

\paragraph{Характерні точки (перетин із осями координат)}
Характерні точки відсутні.

\begin{figure}[h]
	\centering
	\begin{tikzpicture}
	\begin{axis}
		[
			axis lines = middle,
			samples=1000,
			xmin = -.5,
			xmax = 1.5,
		]
		\addplot [violet] gnuplot {1/x**2 - 1/(x-1)**2};
	\end{axis}
	\end{tikzpicture}
	\caption{графік функції
	$y = \frac{1}{x^2} - \frac{1}{(x-1)^2}$
}
\end{figure}

% \Answer{<++>}
\end{document}
