%shell-escape
\documentclass[../rgr1.tex]{subfiles}
\usetikzlibrary{patterns}

\begin{document}

\Problem{
	Дослідити методами диференціального числення функцію $y = 2x - \arcsin x$ та побудувати її графік
}

\Solution

\paragraph{ОВ} $D(\arcsin): x \in [-1;1],~D(2x): x \in \mathbb{R} \to D(f) = [-1;1]$
\paragraph{ОЗ $E(f)$} $f \in ( -1; 1 )$
\paragraph{Парність}
$
	y(-x) = -2x - \arcsin (-x)
	= - \big( 2x - \arcsin (x) \big)
	%\neq
	%2x - \arcsin x
$, функція непарна.

\paragraph{Періодичність $[f(x+T) = f(x)]?$}

\begin{align}
	2(x+T)-\arcsin(x+T) = 2(x)-\arcsin(x) \\
	T = \arcsin(x+T) - \arcsin(x)
\end{align}

% $$ <++>
% 	\arcsin(x) + \arcsin(y) = \arcsin( x * \sqrt{1 - y^2} + y * \sqrt{1 - x^2} )\\
% 	T = \arcsin \left( (x+T)*\sqrt{1-x^2} + x*\sqrt{1-(x+T)^2} \right) \\
% $$

$T$ залежить від $x$, функція неперіодична

\paragraph{Неперервність} Опис дослідження на неперервність наведено в завданні \ref{continuity}, тому тут лише обчислення

Чииииииии функція неперервна на всій області визначення?

\paragraph{Точки екстремуму}
Знайдімо похідну функції.
\begin{equation}
	y = 2x - \arcsin x \to
	f'(x) = 2 - \frac{1}{\sqrt{1-x^2}}
\end{equation}

Точки екстремуму знаходяться в місцях, де похідна функції змінює знак.
$x \neq \pm1$
\begin{equation}
	\frac{1}{\sqrt{1-x^2}} = 2 \to \sqrt{1-x^2} = \frac{1}{2}
	\to 1-x^2 = \frac{1}{4} \to x^2 = \frac{3}{4}
	\to x = \pm \frac{\sqrt 3}{2}
\end{equation}
\begin{align}
	x = 0 \to f'(x) = 1 > 0 \\
	x = \pm \frac{\sqrt{3.5} }{2} \to f'(x)
	= 2 - \frac{1}{\sqrt{1-\frac{3\frac{1}{2}}{4}}}
	= 2 - \frac{1}{\sqrt{1-{\frac{7}{8}}}}
	= 2 - \frac{1}{\sqrt {\frac{1}{8}} } \\
	\sqrt{\frac{1}{8}} = 0.35,~
	2 - \frac{100}{35} \approx -0.8 < 0.
\end{align}

Перенісши значення на числову пряму, матимемо:
\begin{figure}[h]
	\centering
	\begin{tikzpicture}[
		letter/.style={
			circle, minimum size=3pt, inner sep=0,
			outer sep=0, fill=black, label=below:#1
		},
		crossed/.style={
			cross out, minimum size=3pt, inner sep=0,
			outer sep=0, draw=black, label=below:#1
		},
		number/.style={fill=white, draw=red, pos=.5}
	]
		\draw (-1,0) --
			node(A)[crossed=-1,pos=-2]{}
			node(Al) [letter=$-\frac{\sqrt 3}{2}$, pos=-.7]{}
			node(Bl) [letter=$\frac{\sqrt 3}{2}$, pos=.7]{}
			node(B)[crossed=1,pos=2]{}
			(1,0)
		;
		\draw (-5,0) -- (-1,0);
		\draw (3,0) -- (1,0);
		\draw[]
			(Al) to[bend left=50] node[number]{+} (Bl)
			(A) to[bend right=40] node[number]{-} (Al)
			(B) to[bend left=40] node[number]{-} (Bl)
		;
	\end{tikzpicture}
\end{figure}

$x = \frac{\sqrt 3}{2}$ --- максимум,
$x = -\frac{\sqrt 3}{2}$ --- мінімум.

\paragraph{Проміжки монотонності}

Поведінку функції на проміжку можна визначити за знаком похідної:
\subparagraph{Зростання:} $x\in \left[ -\frac{\sqrt 3}{2}; \frac{\sqrt 3}{2} \right]$
\subparagraph{Спадання:} $x\in \left( -1; -\frac{\sqrt 3}{2} \right] \bigcup \left[ \frac{\sqrt 3}{2}; +1 \right)$

\paragraph{Опуклість ($f''$)}

\begin{dmath}
	f''(x) = \left( 2 - \frac{1}{\sqrt{1-x^2}} \right)'
	= \left( - \frac{1}{ (1-x^2)^{\frac{1}{2}} } \right)'
	= - \left( (1-x^2)^{-\frac{1}{2}} \right)' * 2x
	= - \frac{1}{2}*(1-x^2)^{-\frac{3}{2}} * 2x
	= - \frac{\overbrace{\sqrt{(1-x^2)^3}}^{> 0} * 2x}{2}
\end{dmath}

Опуклість визначає знак $-x$.
Отже, функція опукла вгору, коли x > 0, та опукла вниз, коли x < 0.
\paragraph{$\bigcup$} $x \in (-1; 0)$
\paragraph{$\bigcap$} $x \in (0; +1)$

\paragraph{Асимптоти}

Похилих асимптот функція не має, а вертикальні знаходяться в точках,
де $x$ набуває значення $1$ та $-1$.

% \subparagraph{вертикальні:}

% \subparagraph{похилі:}
% $y = kx + b, k = \lim_{x\to\infty}\frac{f(x)}{x},$
% $ b = \lim_{x\to\infty} ( f(x) - kx )$

% \begin{equation}
% 	k = \lim_{x\to\infty}\frac{2x - \arcsin x}{x} =
% 	\lim_{x\to\infty}\frac{2x - \arcsin x}{x} =
% \end{equation}

\paragraph{Характерні точки (перетин із осями координат)}
$O(0; 0)$ --- характерна точка.

%-----------------------------------

\begin{figure}[h]
	\centering
	\begin{tikzpicture}
	\begin{axis}
		[
			axis lines = middle,
			samples=1000,
			xmin = -1,
			xmax = 1
			]
		\addplot [violet] gnuplot {2*x-asin(x)};
		%\addplot [blue] gnuplot {asin(x)};
	\end{axis}
	\end{tikzpicture}
	%\subfile{plot.tex}
	\caption{графік функції $y = 2x - \arcsin x$}
\end{figure}

% \Answer{<++>}
\end{document}
