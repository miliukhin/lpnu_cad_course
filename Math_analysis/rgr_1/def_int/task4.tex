\documentclass[../rgr1.tex]{subfiles}

\begin{document}

\Problem{Обчислити площу фігури, обмеженої лініями, які
	задані параметрично або в полярній системі координат.
}

$x=a(t-\sin t), y=a(1-\cos t), y=0, 0\leq t\leq2\pi$

\Solution

\begin{figure}[h]
% \begin{center}
\centering
  \begin{tikzpicture}
		\begin{axis}
		[
			axis lines = middle,
			% color=white,
			% samples=10000,
			y = 1cm,
			hide axis
		]
		\addplot[red,domain=0:2.5*pi,samples=50, name path=cycloid] plot ({\x - sin(\x r)},{1 - cos(\x r)});
    		\path[name path=axis] (axis cs:0,0) -- (axis cs:4,0);
	\addplot [
		thick,
		pattern color=purple,
		pattern=north west lines,
		fill opacity=0.4
	]
	fill between[
		of=cycloid and axis,
		soft clip={domain=0:2*pi},
    	];
		\end{axis}

  \coordinate (O) at (0,0);
  \coordinate (A) at (0,3);
  \def\r{1} % radius
  \def\c{1.4} % center
  \coordinate (C) at (\c, \r);

  \draw[-latex] (O) -- (A) node[anchor=south] {$y$};
  \draw[-latex] (O) -- (2.6*pi,0) node[anchor=west] {$x$};
  \draw[red,domain=-0.5*pi:2.5*pi,samples=50, line width=1]
       plot ({\x - sin(\x r)},{1 - cos(\x r)});
  \draw[blue, line width=1] (C) circle (\r);
  \draw[] (C) circle (\r);


  % coordinate x
  \def\x{0.4} % coordinate x
  \def\y{0.83} % coordinate y
  \def\xa{0.3} % coordinate x for arc left
  \def\ya{1.2} % coordinate y for arc left
  \coordinate (X) at (\x, 0 );
  \coordinate (Y) at (0, \y );
  \coordinate (XY) at (\x, \y );

  \node[anchor=north] at (X) {$x$} ;

  % draw center of circle
  \draw[fill=blue] (C) circle (1pt);

  % draw radius of the circle
  \draw[] (C) -- node[anchor=south] {\; $a$} (XY);

  % bottom of circle, radius to the bottom
  \coordinate (B) at (\c, 0);
  \draw[] (C) -- (B) node[anchor=north] {$a \, t$};

  % projections of point XY
  \draw[dotted] (XY) -- (X);
  \draw[dotted] (XY) -- (Y) node[anchor=east, xshift=1mm] {$\quad y$};

  % arc theta
  % start arc
  \coordinate (S) at (\c, 0.4);
  \draw[->] (S) arc (-90:-165:0.6);
  \node[xshift=-2mm, yshift=-2mm] at (C) {\scriptsize $t$};

  % arc above
  \coordinate (AA) at (\xa, \ya);
  \draw[-latex, rotate=25] (AA) arc (-220:-260:1.3);

  % arc below
  \def\xb{2.5} % coordinate x for arc bottom
  \def\yb{0.8} % coordinate y for arc bottom
  \coordinate (AB) at (\xb, \yb);
  \draw[-latex, rotate=-10] (AB) arc (-5:-45:1.3);



  % XY dot
  \draw[fill=black] (XY) circle (1pt);


  % top label
  \coordinate (T) at (pi, 2);
  \node[anchor=south] at (T)  {$(\pi a, 2 a )$} ;
  \draw[fill=black] (T) circle (1pt);

  % equations
  \coordinate (E) at ( 4,1.2);
  \coordinate (F) at ( 4,0.9);
  \node[] at (E) {\scriptsize $x=a(t - \sin t)$};
  \node[] at (F) {\scriptsize $y=a(1 - \cos t)$};

  % label 2pi a
  \coordinate (TPA) at (2*pi, 0);
  \node[anchor=north] at (TPA) {$2 \pi a$};


  \end{tikzpicture}
  \caption{задана крива}
  \label{cycloid}
% \end{center}
\end{figure}

\begin{equation}
	S = -\int_\alpha^\beta y(t)x'(t)dt
\end{equation}

циклоїда рухається за годинниковою стрілкою, тому мінус зникає.

\begin{equation}
	x'(t) = a(1-\cos t)
\end{equation}

\begin{dmath}
	S = \int_\alpha^\beta y(t)x'(t)dt
	= \int_0^{2\pi} a(1-\cos t) a(1-\cos t)
	= a^2\int_0^{2\pi} (1-2\cos t+\cos^2 t)
	= a^2\int_0^{2\pi} (1-2\cos t+\frac{1+\cos 2t}{2})
	= a^2\left(t-2\sin t + \frac{1}{2} t + \frac{1}{4}\sin2t \right) \Bigg|_0^{2\pi}
	% = a^2\left(2\pi-2\sin 2\pi + \frac{1}{2} 2\pi + \frac{1}{4}\sin4\pi \right)
	= 3\pi a^2
\end{dmath}

\Answer{
	$
	3\pi a^2
	$
}

\end{document}
end{document}
