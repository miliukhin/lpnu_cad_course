\documentclass[../rgr_2.tex]{subfiles}

\begin{document}

\Problem{Знайти радіус та інтервал збіжності степеневого ряду
$
	\frac{(x+2)^n}{n\ln(n+2)}
$
}
\Solution

\begin{equation}
	\lim_{n\to\infty} \left|
		\frac
		{n \cdot \ln(n+2)\cdot (x+2)^{n+1}}
		{(n+1) \ln(n+3)(x+2)^n}
		\right|
	= \lim_{n\to\infty} \left|
		\frac
		{n \cdot \ln(n+2)}
		{(n+1) \ln(n+3)}
		\right|
		\cdot |x+2|
		% <++>
	= |x+2| \neq 0, x\neq 0
\end{equation}

\begin{tabular}{lc}
	Абсолютно збіжний: &
	$|x+2| < 1 \implies x \in (-3;-1)$ \\

	Розбіжний: &
	$|x+2| > 1 \implies x \notin (-3;-1)$ \\
\end{tabular}

\Answer{
	Ряд абсолютно збіжний на інтервалі
	$x \in (-3;-1)$,
	радіус його збіжності дорівнює 1
}

\end{document}
