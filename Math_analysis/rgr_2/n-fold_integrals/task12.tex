\documentclass[../rgr_2.tex]{subfiles}
%shell-escape

% \let\oldint\iint % Save the original \int command
% \renewcommand{\iint}{\oldint\limits} % Redefine \int to include \limits

\begin{document}

\Problem{
	Обчислити подвійний інтеграл, переходячи до полярної системи координат
	$
		\iint_D \ln(x^2+y^2)\, dx\, dy,
		D: x^2+y^2\geq e^2,
		x^2+y^2\leq e^4
	$
	}
\Solution

\begin{figure}[h]
	\centering
	\begin{tikzpicture}
	\begin{axis}
    		\addplot [domain=0:2*pi, samples=100, name path=e2] ({exp(1)*cos(deg(x))}, {exp(1)*sin(deg(x))});
    		\addplot [domain=0:2*pi, samples=100, name path=e4] ({exp(2)*cos(deg(x))}, {exp(2)*sin(deg(x))});
	\addplot [
		thick,
		pattern color=purple,
		pattern=north west lines,
	]
	fill between[
		of=e2 and e4,
    	];

	\end{axis}
	\end{tikzpicture}
	\caption{}
\end{figure}

\begin{equation}
	y^2 \geq e^2-x^2 \\
	y=\pm\sqrt{e^2-x^2}
	x^2 \leq e^2
\end{equation}

\begin{equation}
	\begin{aligned}
		x^2+y^2=e^2 \implies r^2=e^2 \implies r=e\\
		x^2+y^2=e^4 \implies r=e^2
	\end{aligned}
\end{equation}
\begin{equation}
	\begin{aligned}
		0\leq\phi\leq2\pi\\
		e\leq r\leq e^2\\
	\end{aligned}
\end{equation}

\begin{dmath}
	\iint_D \ln(x^2+y^2)\, dx\, dy
	= \iint_{G_{r\phi}} \ln(\overbrace{r^2\cos^2\phi+r^2\sin^2\phi}^{r^2})r\, dr\, d\phi
	= \iint_{G_{r\phi}} \ln r^3\, dr\, d\phi
	= \int_0^{2\pi}\,d\phi \int_e^{e^2} \ln r^3\, dr
	= \int_0^{2\pi}\,d\phi\, 3\int_e^{e^2} \ln r\, dr
	= \int_0^{2\pi}\,d\phi\, (3r\ln r-3r)\Bigg|_e^{e^2}
	= \int_0^{2\pi}\,d\phi\, (3e^2\ln e^2-3e^2)-(3e\ln e-3e)
	= \int_0^{2\pi}3e^2\,d\phi
	= 3e^2\phi \Bigg|_0^{2\pi}
	= 6\pi e^2
\end{dmath}
\Answer{
	$6\pi e^2$
}

\end{document}
