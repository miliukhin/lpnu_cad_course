%xelatex
\documentclass[14pt]{extreport}

\usepackage[T2A]{fontenc}
\usepackage[utf8]{inputenc}
\usepackage[english,ukrainian]{babel}
\usepackage{amsmath}
\usepackage{mathtools}
\usepackage{caption}
\usepackage{mathspec}
\setallmainfonts{Nimbus Roman}
\usepackage{graphicx}
\usepackage[a4paper,margin=0.5in]{geometry}
\usepackage{pgfplots}
\pgfplotsset{width=10cm,compat=1.8}
\usepackage{indentfirst}
\usepackage{boxedminipage}
\usepackage{wrapfig}
\usepackage{subfiles}

\begin{document}
\pagestyle{empty}

\subfile{title}

\subsection*{Мета роботи}

Мета роботи – ознайомитись на практиці з основними поняттями та
законами логіки висловлювань, навчитись доводити логічні твердження шляхом
побудови таблиць істинності і використовувати закони логіки.

\begin{center}\bf Варіант 12\end{center}
%\begin{itemize}
%	\item $\neg,~\bar p - \text{заперечення};$
%	\item $p \lor q \vee - \text{диз'юнкція (АБО, тобто одне з них правильне то ок)};$
%	\item $p \land q \wedge,~p\&q - \text{кон'юнкція (І, обидва мають бути вірними)};$
%	\item $p \oplus q - \text{альтернативне "або" ($p$ та $q$ мають різні значення,
%		тоді воно правильне)}$
%	\item $	p \rightarrow q - \text{імплікація, хибна тільки тоді, коли $p=1, q=0$}$
%		\\``[...] достатньо для [...]''; ``Якщо p, то q'', але необов'язково p, бо трава буває мокра без дощу
%	\item $	p \sim q - \text{еквівалентність, істинна за однакових значень.
%	``тоді й лише тоді''}$
%\end{itemize}
%
%n-місна формула – це формула, що містить n атомів.
%Вона має 2n інтерпретацій (наборів), тобто існує 2n
%способів надати значення істинності її атомам.
%
%\smallskip
%
%Виконувана (здійснення) формула – це формула, яка має
%принаймні одну інтерпретацію, у якій вона набуває значення істинності.
%
%\begin{itemize}
%	\item Тавтологія – формула, що виконується у всіх інтерпретаціях.
%
%	\item Протиріччя – формула, що не виконується у жодній інтерпретації.
%В інших випадках формулу називають ані істинною, ані хибною.
%\end{itemize}
%
%\subsubsection*{Закони логіки висловлювань у визначенні
%еквівалентності формул}
%
%\begin{enumerate}
%	\item Вилучити імплікації, еквівалентонсті та альтернативне або:\begin{itemize}
%			\item $p\sim q = (p\rightarrow q)\land(q\rightarrow p);$
%			\item $(p\rightarrow q) = \neg p \lor q;$
%			\item $p \oplus q = \bar{p\sim q}.$
%		\end{itemize}
%	\item Позбутися знаків заперечень над великими виразами (де Морган, подвійне заперечення)
%	\item Спростити.
%\end{enumerate}
%
\subsubsection*{1. Записати у вигляді формули дане твердження, позначаючи прості
висловлення буквами.}

Чотирикутник $ABCD$~--- квадрат тоді, коли він є прямокутником.
%$\widehat{\dbinom{\odot_\text{v}\odot}{\wr}}$

Розіб'ємо висловлювання на атоми і позначимо їх:
\begin{itemize}
	\item ``$ABCD$ є квадратом''~--- $p$,
	\item ``$ABCD$ є прямокутником''~--- $q$.
\end{itemize}
$$p\to q$$

\subsubsection*{2. Побудувати таблицю істинності для висловлювання}
$(a\rightarrow b)\rightarrow((a\land b)\rightarrow b)$

\medskip

\begin{tabular}{|c|c|c|c|c|c|}
	\hline
	$a$ & $b$ & $a\rightarrow b$ & $(a\land b)$ &
	$(a\land b)\rightarrow b$ &
	$(a\rightarrow b)\rightarrow((a\land b)\rightarrow b)$\\
	\hline
	0 & 0 & 1 & 0 & 1 & 1 \\
	\hline
	0 & 1 & 1 & 0 & 1 & 1 \\
	\hline
	1 & 0 & 0 & 0 & 1 & 1 \\
	\hline
	1 & 1 & 1 & 1 & 1 & 1 \\
	\hline
\end{tabular}

\medskip
Висловлювання виявилося тавтологією.

\subsubsection*{3. Побудовою таблиць істинності вияснити чи є висловлювання тавтологією чи запереченням чи виконуваною формулою:}

\begin{center}
\begin{tabular}{|c|c|c|c|c|c|c|}
	\hline
	$p$ & $q$ & $\bar q$ & $p\to q=a$ & $q\to \bar q = b$
	& $a \land b$ & $(a\land b)\sim p $\\
	\hline
	0 & 0 & 1 & 1 & 1 & 1 & 0 \\
	\hline
	0 & 1 & 0 & 1 & 0 & 0 & 1 \\
	\hline
	1 & 0 & 1 & 0 & 1 & 0 & 0 \\
	\hline
	1 & 1 & 0 & 1 & 0 & 0 & 0 \\
	\hline
\end{tabular}
\end{center}

$$
((p\to q)\land(q\to \bar q))\sim p =
((\bar p \lor q) \land (\bar q \lor \bar q))\sim p =
$$
$$
(((\bar p \lor q) \land \bar q)\to p) \land
(p \to ((\bar p \lor q) \land \bar q))=
$$
$$
(\overline{(\bar p \lor q) \land \bar q)}\lor p) \land
(\bar p \lor ((\bar p \lor q)\land \bar q))=
$$
$$
(\overline{(\bar p \lor q)} \lor q)\lor p) \land
(\bar p \lor ((\bar p \land \bar q)\lor (q \land \bar q))=
$$
$$
((p \land \bar q) \lor q)\lor p) \land
(\bar p \lor ((\bar p \land \bar q)\lor 0)=
$$
$$
(((p\lor q) \land (\bar q \lor q))\lor p) \land
(\bar p \lor (\bar p \land \bar q)=
$$
$$
(((p\lor q) \land 1)\lor p) \land
\bar p =
$$
$$
((p\lor q) \lor p) \land
\bar p =
(p\lor q) \land \bar p=
$$
$$
(p\land \bar p) \lor (q\land \bar p)=0\lor(q\land\bar p).
$$

Висловлювання є виконуваною формулою.

\subsection*{Висновок}

Виконавши цю практичну роботу, я закріпив свої знання
основ математичної логіки, зокрема логічних операцій
та побудови таблиць істинності.

\end{document}
