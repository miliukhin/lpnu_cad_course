%xelatex
\documentclass[14pt]{extreport}

\usepackage[T2A]{fontenc}
\usepackage[utf8]{inputenc}
\usepackage[english,ukrainian]{babel}
%\usepackage{fontspec}
\usepackage{amsmath}
%\usepackage{mathtools}
\usepackage{caption}
\usepackage{mathspec}
\setallmainfonts{Nimbus Roman}
\usepackage{graphicx}
\usepackage[a4paper,margin=0.5in]{geometry}
\usepackage{indentfirst}
\usepackage{tikz}
\usetikzlibrary{calc,positioning}
\usepackage{subfiles}

\begin{document}
\pagestyle{empty}

\subfile{title}

\subsubsection*{Мета роботи}

Мета роботи – ознайомитись на практиці з основними поняттями
мереж, потоків у мережах, навчитись знаходити максимальний потік у
мережі.

\begin{center}\bf Варіант 12\end{center}

\subsection*{Побудувати повний потік,
а потім скорегувати його до найбільшого
(дуги спрямовані зліва направо)}
\begin{figure}[h]
	\subfile{0.tex}
\end{figure}

\subsection*{Побудова повного потоку}

Покрокова побудова
повного потоку
відображена на
рис. \ref{startfull}-\ref{endfull}.
Його значення:

$$V=14+4+6+1+3+2+1+3+2=36.$$

%	\subfile{1.tex}
%
%	\subfile{2.tex}
%
%	\subfile{3.tex}
%
%	\subfile{4.tex}
%
%	\subfile{5.tex}
%
%	\subfile{6.tex}
%
%	\subfile{7.tex}
%
%	\subfile{8.tex}
%
%	\subfile{9.tex}

\begin{figure}[h]
	\centering
	\subfile{1.tex}
	\caption{\Delta=14}
	\label{startfull}
\end{figure}
\begin{figure}[h]
	\centering
	\subfile{2.tex}
	\caption{\Delta=4}
\end{figure}
\begin{figure}[h]
	\centering
	\subfile{3.tex}
	\caption{\Delta=6}
\end{figure}
\begin{figure}[h]
	\centering
	\subfile{4.tex}
	\caption{\Delta=1}
\end{figure}
\begin{figure}[h]
	\centering
	\subfile{5.tex}
	\caption{\Delta=3}
\end{figure}
\begin{figure}[h]
	\centering
	\subfile{6.tex}
	\caption{\Delta=2}
\end{figure}
\begin{figure}[h]
	\centering
	\subfile{7.tex}
	\caption{\Delta=1}
\end{figure}
\begin{figure}[h]
	\centering
	\subfile{8.tex}
	\caption{\Delta=3}
\end{figure}
\begin{figure}[h]
	\centering
	\subfile{9.tex}
	\caption{\Delta=2}
	\label{endfull}
\end{figure}


\subsection*{Побудова максимального потоку}

Коригуючий шлях для
максимального потоку
зображений на рис. \ref{max}.
Величина максимального потоку~---41.


\begin{figure}[h]
	\centering
	\subfile{corr.tex}
	\caption{\Delta=5}
	\label{max}
\end{figure}

\subsection*{Висновок}

Виконання цієї практичної роботи дозволило
мені краще зрозуміти алгоритм побудови повного
та максимального потоків у мережах.

В заданому графі не було чітко видно вагу ребра 5-6, тому я обрав її як 4.

\end{document}
