\documentclass[14pt]{extreport}

\usepackage[T2A]{fontenc}
\usepackage[utf8]{inputenc}
\usepackage[english,ukrainian]{babel}
\usepackage{caption}
\usepackage{graphicx}
\usepackage{tikz}
\usetikzlibrary{tikzmark}
\usepackage[a4paper,margin=0.5in]{geometry}
\usepackage{indentfirst}
\usepackage{subfiles}

\begin{document}
\subsection*{Відповіді на контрольні запитання}

\subsubsection{1. У чому полягає мінімізація булевих функцій?}

Мінімізація булевих функцій полягає у пошуку
їх найпростіших форм.

\subsubsection*{2. Дайте визначення мінімальної ДНФ?}

Мінімальна ДНФ~--- ДНФ, що складається з
якнайменшої кількості букв (беруть до уваги
всі їх появи у функції).

\subsubsection*{3. Що таке проста імпліканта булевої функції?}

Проста імпліканта~--- це імпліканта, яка
при вилученні довільної букви перестає бути
імплікантою.

\subsubsection*{4. Що таке скорочена ДНФ?}

Скорочена ДНФ~--- ДНФ, що складається
з усіх простих імплікант.

\subsubsection*{5. Що таке тупикова ДНФ?}

Тупикова ДНФ функції~--- це ДНФ, у якій усі
імпліканти прості та при вилученні одної з
них утвориться ДНФ, що не відповідає функції.

\subsubsection*{6. Назвіть етапи знаходження мінімальної ДНФ?}

\begin{enumerate}
	\item Побудова СДНФ;
	\item Побудова тупикових ДНФ та вибір із них мінімальних.
\end{enumerate}

\subsubsection*{7. Які кроки виконують до ДДНФ у методі Куайна побудови МДНФ?}


\begin{enumerate}
	\item Для побудови СДНФ:
	\begin{enumerate}
		\item Запис функції у ДДНФ;
		\item Застосування до її членів неповного
			склеювання та поглинання.
	\end{enumerate}
	\item Для пошуку мінімальних ДНФ~---
		Побудова імплікантної таблиці Куайна:
	\begin{enumerate}
		\item Нанесення простих імплікант функції як
			рядків та комбінацій значень змінних, за яких
			функція набуває правдивого значення як
			стовпців;
		\item Позначення клітинок, де проста імпліканта
			(частково або повністю) збігається з
			набором значень змінних;
		\item У стовпцях із лише одною позначкою
			відповідну імпліканту вносять до
			імплікант ядра (вона має входити
			у будь-яку ТДНФ);
		\item Викреслення рядків, що відповідають
			імплікантам ядра та стовпців, у
			яких хоч одна позначена клітинка
			викреслена;
		\item Виділення імплікант ядра та імплікант,
			що відповідають невикресленим клітинкам
			та запис отриманих тупикових ДНФ.
	\end{enumerate}
\end{enumerate}


\subsubsection*{8. Що є основним апаратом для побудови тупикової ДНФ у методі Куайна?}

Основним апаратом для побудови тупикової ДНФ у методі Куайна є імплікантна таблиця.

\subsubsection*{9. Що таке функціональна схема, функціональний елемент?}

Функціональна схема~--- це інтерпретація булевої функції,
	яку можна реалізувати як електронний пристрій. Функціональний
	елемент~--- це частина схеми, що реалізує певну логічну операцію.

\subsubsection*{10. Які функціональні елементи складають повну систему для побудови
функціональної схеми?}

Повну систему для побудови функціональної схеми складають схема логічного
додавання та інвертор. У деяких ситуаціях можна їх замінити на елемент, що
відповідає операції НЕ-І.


\end{document}
