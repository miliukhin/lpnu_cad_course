%-shell-escape, якщо використовуєте minted
\documentclass[a4paper, 12pt, oneside]{extarticle}
\input{$HOME/Templates/lpnu_doc_templates/settings/preamble.tex}
% якщо домахуються за Times New Roman, то
% використовуєте xelatex і цей файл:
\input{$HOME/Templates/lpnu_doc_templates/settings/font_styles.tex}
\input{$HOME/Templates/lpnu_doc_templates/settings/minted_settings.tex}

\newcommand\Variant{4}
\newcommand\Date{22.04.\the\year}
\newcommand\Discipline{Алгоритмізація та програмування, частина 2}
\newcommand\Instructor{Кулешник Я. Ф.}

\newcommand\Type{\Lab}
\newcommand\Number{5}
\newcommand\Topic{ДИНАМІЧНІ ОБ’ЄКТИ СКЛАДНОЇ СТРУКТУРИ.
БІНАРНІ ДЕРЕВА ПОШУКУ}

\begin{document}
\Margins

\Margins
%\begin{wrapfigure}[3]{l}{.27\textwidth}
%\includegraphics[width=.28\textwidth]{$UNI/.templates/lpnu_logo.png}
%\end{wrapfigure}

%\noindent\textbf{Прізвище:} \Lname \\
%\noindent\textbf{Ім'я:} \Fname \\
%\noindent\textbf{Група:} \Group \\
%\noindent\textbf{Варіант:} \Variant \\
%\noindent\textbf{Дата захисту:} \Date \\
%\\
%\noindent\textbf{Кафедра:} \Department \\
%\noindent\textbf{Дисципліна:} \Discipline \\
%\noindent\textbf{Перевірив:} \Instructor \\

%%\medskip\bigskip

%\begin{center}
%	\textbf{ЗВІТ}		\\
%	до \Type~\No\Number	\\
%	на тему ``\Topic''	\\
%\end{center}

% \begin{table}
%   \begin{tabularx}{\textwidth}{|c|X|X|}
%     \hline
%     % Image & Content & Additional Info \\
%     % \hline
% 	  \multirow{3}{*}{\includegraphics[width=4cm]{$UNI/.templates/lpnu_logo.png}}
% 	  & \textbf{ЗВО:}
% 	  Національний університет ``Львівська Політехніка''.
% 	  & \textbf{Тема:}
% 	  \Topic
% 	  \\
% 	  & \textbf{Навчальний рік:}
% 	  2023/2024
% 	  & \textbf{Інститут}
% 	  комп'ютерних наук та інформаційних технологій
% 	  \\
% 	  & \textbf{Семестр:}
% 	  осінній
% 	  & \textbf{Група:}
% 	  \Group
% 	  \\
% 	  & \textbf{Навчальна дисципліна:}
% 	  \Discipline
% 	  & \textbf{Студент:}
% 	  Мілюхін Олександр
% 	  \\
% 	  & \textbf{Кафедра}
% 	  систем автоматизованого проектування
% 	  &
% 	  \\
% 	  & \textbf{Викладач:}
% 	  Чумакевич В. В.
% 	  &
% 	  \\
%     \hline
%   \end{tabularx}
% \end{table}

\setlength{\textfloatsep}{-16pt}
% \setlength{\intextsep}{0pt}

\begin{table}
	\begin{tabular}{|l|l|p{6cm}|}
    \hline
    % Image & Content & Additional Info \\
    % \hline
	  \makecell[l]{
	  \includegraphics[width=3.37cm]{$UNI/.templates/lpnu_logo.png}
  }
	  & \makecell[l]{
	  \textbf{ЗВО:}
	  Національний університет \\ ``Львівська Політехніка''.
	  \\
	  \textbf{Навчальний рік:}
	  2023/2024
	  \\
	  \textbf{Семестр:}
	  осінній
	  \\
	  \textbf{Навчальна дисципліна:} \\
	  \Discipline
	  \\
	  \textbf{Кафедра}
	  систем автоматизованого \\ проектування
	  \\
	  \textbf{Викладач:}
	  Чумакевич В. В.
}
	  & \makecell [l] {
	  \textbf{Тема:}
	  \Topic
	  \\
          \textbf{Інститут}
	  комп'ютерних наук та \\ інформаційних технологій
	  \\
	  \textbf{Група:}
	  \Group
	  \\
	  \textbf{Студент:}
	  Мілюхін Олександр
  }
  \\
    \hline
  \end{tabular}
\end{table}
\section{Мета роботи}

% \begin{table}
%   \begin{tabularx}{\textwidth}{|p{6cm}|c|c|}
% 	  \hline
%     \multirow{3}{*}{\includegraphics[width=6cm]{$UNI/.templates/lpnu_logo.png}}
% 	  & ЗВО: Національний університет ``Львівська Політехніка''
% 	  & Additional Info 1 \\
%     & Content 2 & Additional Info 2 \\
%     & Content 3 & Additional Info 3 \\
% 	  \hline
%   \end{tabularx}
% \end{table}

% \begin{table}
%   \begin{tabular}{|c|c|c|}
%     \hline
%     \multirow{3}{*}{\includegraphics[width=3cm]{$UNI/.templates/lpnu_logo.png}} & \makecell{Content 1 \\ Content 2 \\ Content 3} & \makecell{Additional Info 1 \\ Additional Info 2 \\ Additional Info 3} \\
%     \hline
%   \end{tabular}
% \end{table}


У даній лаборатній буде розглянуто абсолютно нову структуру даних - дерево.
А точніше, двійкове (бінарне) дерево пошуку (binary search tree).

\section*{Індивідуальне завдання}

Сформувати дерево виведення арифметичного виразу a+b*(b-d). Додати вузол для
виразу: (-a)+b*(b-d). Вивести на друк обидва дерева.

\section*{Етапи розв'язку}

\subsection*{Код програми}

\inputminted{c++}{/home/sasha/Documents/uni/2-сем/AlgoProgr/lab5/tree.cpp}

\subsection*{Результат виконання програми}

\begin{verbatim}
[sasha@honeypot ~uni/AlgoProgr/lab5]$ ./tree
.........d
......-
.........b
...*
......b
+
...a
.........d
......-
.........b
...*
......b
+
...a
......-
\end{verbatim}

\section*{Висновок}

Ця лабораторна робота дала мені базове розуміння бінарних дерев
та їх реалізації мовою c++.

\section*{Відповіді на контрольні запитання}
\begin{itemize}
	\question Що таке бінарне дерево?
	\answer Бінарне дерево - це структура даних, що складається з вузлів, де кожен вузол може мати не більше двох дочірніх вузлів, відомих як "лівий" та "правий" дочірні вузли. Ця структура даних використовується для зберігання та пошуку даних.

	\question Які поля містить динамічний об’єкт, що описує бінарне дерево?
	\answer Поле з даними, вказівники на лівий та правий дочірні вузли

	\question Що може бути ключем у бінарному дереві?
	\answer Будь-який типом даних, які можна порівнювати між собою, наприклад, числом, рядком, символом, об'єктом тощо. Важливо, щоб ключ був унікальним для кожного елемента в дереві, щоб можна було швидко знайти та доступатися до потрібного елемента.

	\question Як формується бінарне дерево?
	\answer Бінарне дерево формується за допомогою поступового додавання елементів. Починаючи з кореня дерева, новий елемент порівнюється з ключем кореня. Якщо ключ нового елемента менший за ключ кореня, то елемент додається до лівого піддерева, інакше - до правого. Для кожного наступного елемента порівнювання проводиться з ключами вже існуючих елементів піддерева, до якого додавався попередній елемент. Цей процес продовжується доти, поки всі елементи не будуть додані до дерева.

	\question Яка структура бінарного дерева, що таке вершина або корінь дерева, що таке гілки дерева?
	\answer Бінарне дерево складається з вершин та гілок. Вершина - це елемент дерева, який містить ключ та посилання на ліве та праве піддерево (якщо вони існують). Корінь дерева – це вершина, з якої починається дерево, тобто вершина без батьківського елемента. Гілка - це зв'язок між вершинами, який вказує на те, який елемент має бути доданий до якого піддерева при додаванні нового елемента. В бінарному дереві кожна вершина може мати не більше двох нащадків - лівий та правий сини.
\end{itemize}

\end{document}
